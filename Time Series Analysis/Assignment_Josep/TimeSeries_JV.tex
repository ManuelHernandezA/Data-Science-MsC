\documentclass[11pt]{article}

    \usepackage[breakable]{tcolorbox}
    \usepackage{parskip} % Stop auto-indenting (to mimic markdown behaviour)
    

    % Basic figure setup, for now with no caption control since it's done
    % automatically by Pandoc (which extracts ![](path) syntax from Markdown).
    \usepackage{graphicx}
    % Maintain compatibility with old templates. Remove in nbconvert 6.0
    \let\Oldincludegraphics\includegraphics
    % Ensure that by default, figures have no caption (until we provide a
    % proper Figure object with a Caption API and a way to capture that
    % in the conversion process - todo).
    \usepackage{caption}
    \DeclareCaptionFormat{nocaption}{}
    \captionsetup{format=nocaption,aboveskip=0pt,belowskip=0pt}

    \usepackage{float}
    \floatplacement{figure}{H} % forces figures to be placed at the correct location
    \usepackage{xcolor} % Allow colors to be defined
    \usepackage{enumerate} % Needed for markdown enumerations to work
    \usepackage{geometry} % Used to adjust the document margins
    \usepackage{amsmath} % Equations
    \usepackage{amssymb} % Equations
    \usepackage{textcomp} % defines textquotesingle
    % Hack from http://tex.stackexchange.com/a/47451/13684:
    \AtBeginDocument{%
        \def\PYZsq{\textquotesingle}% Upright quotes in Pygmentized code
    }
    \usepackage{upquote} % Upright quotes for verbatim code
    \usepackage{eurosym} % defines \euro

    \usepackage{iftex}
    \ifPDFTeX
        \usepackage[T1]{fontenc}
        \IfFileExists{alphabeta.sty}{
              \usepackage{alphabeta}
          }{
              \usepackage[mathletters]{ucs}
              \usepackage[utf8x]{inputenc}
          }
    \else
        \usepackage{fontspec}
        \usepackage{unicode-math}
    \fi

    \usepackage{fancyvrb} % verbatim replacement that allows latex
    \usepackage{grffile} % extends the file name processing of package graphics 
                         % to support a larger range
    \makeatletter % fix for old versions of grffile with XeLaTeX
    \@ifpackagelater{grffile}{2019/11/01}
    {
      % Do nothing on new versions
    }
    {
      \def\Gread@@xetex#1{%
        \IfFileExists{"\Gin@base".bb}%
        {\Gread@eps{\Gin@base.bb}}%
        {\Gread@@xetex@aux#1}%
      }
    }
    \makeatother
    \usepackage[Export]{adjustbox} % Used to constrain images to a maximum size
    \adjustboxset{max size={0.9\linewidth}{0.9\paperheight}}

    % The hyperref package gives us a pdf with properly built
    % internal navigation ('pdf bookmarks' for the table of contents,
    % internal cross-reference links, web links for URLs, etc.)
    \usepackage{hyperref}
    % The default LaTeX title has an obnoxious amount of whitespace. By default,
    % titling removes some of it. It also provides customization options.
    \usepackage{titling}
    \usepackage{longtable} % longtable support required by pandoc >1.10
    \usepackage{booktabs}  % table support for pandoc > 1.12.2
    \usepackage{array}     % table support for pandoc >= 2.11.3
    \usepackage{calc}      % table minipage width calculation for pandoc >= 2.11.1
    \usepackage[inline]{enumitem} % IRkernel/repr support (it uses the enumerate* environment)
    \usepackage[normalem]{ulem} % ulem is needed to support strikethroughs (\sout)
                                % normalem makes italics be italics, not underlines
    \usepackage{mathrsfs}
    

    
    % Colors for the hyperref package
    \definecolor{urlcolor}{rgb}{0,.145,.698}
    \definecolor{linkcolor}{rgb}{.71,0.21,0.01}
    \definecolor{citecolor}{rgb}{.12,.54,.11}

    % ANSI colors
    \definecolor{ansi-black}{HTML}{3E424D}
    \definecolor{ansi-black-intense}{HTML}{282C36}
    \definecolor{ansi-red}{HTML}{E75C58}
    \definecolor{ansi-red-intense}{HTML}{B22B31}
    \definecolor{ansi-green}{HTML}{00A250}
    \definecolor{ansi-green-intense}{HTML}{007427}
    \definecolor{ansi-yellow}{HTML}{DDB62B}
    \definecolor{ansi-yellow-intense}{HTML}{B27D12}
    \definecolor{ansi-blue}{HTML}{208FFB}
    \definecolor{ansi-blue-intense}{HTML}{0065CA}
    \definecolor{ansi-magenta}{HTML}{D160C4}
    \definecolor{ansi-magenta-intense}{HTML}{A03196}
    \definecolor{ansi-cyan}{HTML}{60C6C8}
    \definecolor{ansi-cyan-intense}{HTML}{258F8F}
    \definecolor{ansi-white}{HTML}{C5C1B4}
    \definecolor{ansi-white-intense}{HTML}{A1A6B2}
    \definecolor{ansi-default-inverse-fg}{HTML}{FFFFFF}
    \definecolor{ansi-default-inverse-bg}{HTML}{000000}

    % common color for the border for error outputs.
    \definecolor{outerrorbackground}{HTML}{FFDFDF}

    % commands and environments needed by pandoc snippets
    % extracted from the output of `pandoc -s`
    \providecommand{\tightlist}{%
      \setlength{\itemsep}{0pt}\setlength{\parskip}{0pt}}
    \DefineVerbatimEnvironment{Highlighting}{Verbatim}{commandchars=\\\{\}}
    % Add ',fontsize=\small' for more characters per line
    \newenvironment{Shaded}{}{}
    \newcommand{\KeywordTok}[1]{\textcolor[rgb]{0.00,0.44,0.13}{\textbf{{#1}}}}
    \newcommand{\DataTypeTok}[1]{\textcolor[rgb]{0.56,0.13,0.00}{{#1}}}
    \newcommand{\DecValTok}[1]{\textcolor[rgb]{0.25,0.63,0.44}{{#1}}}
    \newcommand{\BaseNTok}[1]{\textcolor[rgb]{0.25,0.63,0.44}{{#1}}}
    \newcommand{\FloatTok}[1]{\textcolor[rgb]{0.25,0.63,0.44}{{#1}}}
    \newcommand{\CharTok}[1]{\textcolor[rgb]{0.25,0.44,0.63}{{#1}}}
    \newcommand{\StringTok}[1]{\textcolor[rgb]{0.25,0.44,0.63}{{#1}}}
    \newcommand{\CommentTok}[1]{\textcolor[rgb]{0.38,0.63,0.69}{\textit{{#1}}}}
    \newcommand{\OtherTok}[1]{\textcolor[rgb]{0.00,0.44,0.13}{{#1}}}
    \newcommand{\AlertTok}[1]{\textcolor[rgb]{1.00,0.00,0.00}{\textbf{{#1}}}}
    \newcommand{\FunctionTok}[1]{\textcolor[rgb]{0.02,0.16,0.49}{{#1}}}
    \newcommand{\RegionMarkerTok}[1]{{#1}}
    \newcommand{\ErrorTok}[1]{\textcolor[rgb]{1.00,0.00,0.00}{\textbf{{#1}}}}
    \newcommand{\NormalTok}[1]{{#1}}
    
    % Additional commands for more recent versions of Pandoc
    \newcommand{\ConstantTok}[1]{\textcolor[rgb]{0.53,0.00,0.00}{{#1}}}
    \newcommand{\SpecialCharTok}[1]{\textcolor[rgb]{0.25,0.44,0.63}{{#1}}}
    \newcommand{\VerbatimStringTok}[1]{\textcolor[rgb]{0.25,0.44,0.63}{{#1}}}
    \newcommand{\SpecialStringTok}[1]{\textcolor[rgb]{0.73,0.40,0.53}{{#1}}}
    \newcommand{\ImportTok}[1]{{#1}}
    \newcommand{\DocumentationTok}[1]{\textcolor[rgb]{0.73,0.13,0.13}{\textit{{#1}}}}
    \newcommand{\AnnotationTok}[1]{\textcolor[rgb]{0.38,0.63,0.69}{\textbf{\textit{{#1}}}}}
    \newcommand{\CommentVarTok}[1]{\textcolor[rgb]{0.38,0.63,0.69}{\textbf{\textit{{#1}}}}}
    \newcommand{\VariableTok}[1]{\textcolor[rgb]{0.10,0.09,0.49}{{#1}}}
    \newcommand{\ControlFlowTok}[1]{\textcolor[rgb]{0.00,0.44,0.13}{\textbf{{#1}}}}
    \newcommand{\OperatorTok}[1]{\textcolor[rgb]{0.40,0.40,0.40}{{#1}}}
    \newcommand{\BuiltInTok}[1]{{#1}}
    \newcommand{\ExtensionTok}[1]{{#1}}
    \newcommand{\PreprocessorTok}[1]{\textcolor[rgb]{0.74,0.48,0.00}{{#1}}}
    \newcommand{\AttributeTok}[1]{\textcolor[rgb]{0.49,0.56,0.16}{{#1}}}
    \newcommand{\InformationTok}[1]{\textcolor[rgb]{0.38,0.63,0.69}{\textbf{\textit{{#1}}}}}
    \newcommand{\WarningTok}[1]{\textcolor[rgb]{0.38,0.63,0.69}{\textbf{\textit{{#1}}}}}
    
    
    % Define a nice break command that doesn't care if a line doesn't already
    % exist.
    \def\br{\hspace*{\fill} \\* }
    % Math Jax compatibility definitions
    \def\gt{>}
    \def\lt{<}
    \let\Oldtex\TeX
    \let\Oldlatex\LaTeX
    \renewcommand{\TeX}{\textrm{\Oldtex}}
    \renewcommand{\LaTeX}{\textrm{\Oldlatex}}
    % Document parameters
    % Document title
    \title{TimeSeries\_JV}
    
    
    
    
    
% Pygments definitions
\makeatletter
\def\PY@reset{\let\PY@it=\relax \let\PY@bf=\relax%
    \let\PY@ul=\relax \let\PY@tc=\relax%
    \let\PY@bc=\relax \let\PY@ff=\relax}
\def\PY@tok#1{\csname PY@tok@#1\endcsname}
\def\PY@toks#1+{\ifx\relax#1\empty\else%
    \PY@tok{#1}\expandafter\PY@toks\fi}
\def\PY@do#1{\PY@bc{\PY@tc{\PY@ul{%
    \PY@it{\PY@bf{\PY@ff{#1}}}}}}}
\def\PY#1#2{\PY@reset\PY@toks#1+\relax+\PY@do{#2}}

\@namedef{PY@tok@w}{\def\PY@tc##1{\textcolor[rgb]{0.73,0.73,0.73}{##1}}}
\@namedef{PY@tok@c}{\let\PY@it=\textit\def\PY@tc##1{\textcolor[rgb]{0.24,0.48,0.48}{##1}}}
\@namedef{PY@tok@cp}{\def\PY@tc##1{\textcolor[rgb]{0.61,0.40,0.00}{##1}}}
\@namedef{PY@tok@k}{\let\PY@bf=\textbf\def\PY@tc##1{\textcolor[rgb]{0.00,0.50,0.00}{##1}}}
\@namedef{PY@tok@kp}{\def\PY@tc##1{\textcolor[rgb]{0.00,0.50,0.00}{##1}}}
\@namedef{PY@tok@kt}{\def\PY@tc##1{\textcolor[rgb]{0.69,0.00,0.25}{##1}}}
\@namedef{PY@tok@o}{\def\PY@tc##1{\textcolor[rgb]{0.40,0.40,0.40}{##1}}}
\@namedef{PY@tok@ow}{\let\PY@bf=\textbf\def\PY@tc##1{\textcolor[rgb]{0.67,0.13,1.00}{##1}}}
\@namedef{PY@tok@nb}{\def\PY@tc##1{\textcolor[rgb]{0.00,0.50,0.00}{##1}}}
\@namedef{PY@tok@nf}{\def\PY@tc##1{\textcolor[rgb]{0.00,0.00,1.00}{##1}}}
\@namedef{PY@tok@nc}{\let\PY@bf=\textbf\def\PY@tc##1{\textcolor[rgb]{0.00,0.00,1.00}{##1}}}
\@namedef{PY@tok@nn}{\let\PY@bf=\textbf\def\PY@tc##1{\textcolor[rgb]{0.00,0.00,1.00}{##1}}}
\@namedef{PY@tok@ne}{\let\PY@bf=\textbf\def\PY@tc##1{\textcolor[rgb]{0.80,0.25,0.22}{##1}}}
\@namedef{PY@tok@nv}{\def\PY@tc##1{\textcolor[rgb]{0.10,0.09,0.49}{##1}}}
\@namedef{PY@tok@no}{\def\PY@tc##1{\textcolor[rgb]{0.53,0.00,0.00}{##1}}}
\@namedef{PY@tok@nl}{\def\PY@tc##1{\textcolor[rgb]{0.46,0.46,0.00}{##1}}}
\@namedef{PY@tok@ni}{\let\PY@bf=\textbf\def\PY@tc##1{\textcolor[rgb]{0.44,0.44,0.44}{##1}}}
\@namedef{PY@tok@na}{\def\PY@tc##1{\textcolor[rgb]{0.41,0.47,0.13}{##1}}}
\@namedef{PY@tok@nt}{\let\PY@bf=\textbf\def\PY@tc##1{\textcolor[rgb]{0.00,0.50,0.00}{##1}}}
\@namedef{PY@tok@nd}{\def\PY@tc##1{\textcolor[rgb]{0.67,0.13,1.00}{##1}}}
\@namedef{PY@tok@s}{\def\PY@tc##1{\textcolor[rgb]{0.73,0.13,0.13}{##1}}}
\@namedef{PY@tok@sd}{\let\PY@it=\textit\def\PY@tc##1{\textcolor[rgb]{0.73,0.13,0.13}{##1}}}
\@namedef{PY@tok@si}{\let\PY@bf=\textbf\def\PY@tc##1{\textcolor[rgb]{0.64,0.35,0.47}{##1}}}
\@namedef{PY@tok@se}{\let\PY@bf=\textbf\def\PY@tc##1{\textcolor[rgb]{0.67,0.36,0.12}{##1}}}
\@namedef{PY@tok@sr}{\def\PY@tc##1{\textcolor[rgb]{0.64,0.35,0.47}{##1}}}
\@namedef{PY@tok@ss}{\def\PY@tc##1{\textcolor[rgb]{0.10,0.09,0.49}{##1}}}
\@namedef{PY@tok@sx}{\def\PY@tc##1{\textcolor[rgb]{0.00,0.50,0.00}{##1}}}
\@namedef{PY@tok@m}{\def\PY@tc##1{\textcolor[rgb]{0.40,0.40,0.40}{##1}}}
\@namedef{PY@tok@gh}{\let\PY@bf=\textbf\def\PY@tc##1{\textcolor[rgb]{0.00,0.00,0.50}{##1}}}
\@namedef{PY@tok@gu}{\let\PY@bf=\textbf\def\PY@tc##1{\textcolor[rgb]{0.50,0.00,0.50}{##1}}}
\@namedef{PY@tok@gd}{\def\PY@tc##1{\textcolor[rgb]{0.63,0.00,0.00}{##1}}}
\@namedef{PY@tok@gi}{\def\PY@tc##1{\textcolor[rgb]{0.00,0.52,0.00}{##1}}}
\@namedef{PY@tok@gr}{\def\PY@tc##1{\textcolor[rgb]{0.89,0.00,0.00}{##1}}}
\@namedef{PY@tok@ge}{\let\PY@it=\textit}
\@namedef{PY@tok@gs}{\let\PY@bf=\textbf}
\@namedef{PY@tok@gp}{\let\PY@bf=\textbf\def\PY@tc##1{\textcolor[rgb]{0.00,0.00,0.50}{##1}}}
\@namedef{PY@tok@go}{\def\PY@tc##1{\textcolor[rgb]{0.44,0.44,0.44}{##1}}}
\@namedef{PY@tok@gt}{\def\PY@tc##1{\textcolor[rgb]{0.00,0.27,0.87}{##1}}}
\@namedef{PY@tok@err}{\def\PY@bc##1{{\setlength{\fboxsep}{\string -\fboxrule}\fcolorbox[rgb]{1.00,0.00,0.00}{1,1,1}{\strut ##1}}}}
\@namedef{PY@tok@kc}{\let\PY@bf=\textbf\def\PY@tc##1{\textcolor[rgb]{0.00,0.50,0.00}{##1}}}
\@namedef{PY@tok@kd}{\let\PY@bf=\textbf\def\PY@tc##1{\textcolor[rgb]{0.00,0.50,0.00}{##1}}}
\@namedef{PY@tok@kn}{\let\PY@bf=\textbf\def\PY@tc##1{\textcolor[rgb]{0.00,0.50,0.00}{##1}}}
\@namedef{PY@tok@kr}{\let\PY@bf=\textbf\def\PY@tc##1{\textcolor[rgb]{0.00,0.50,0.00}{##1}}}
\@namedef{PY@tok@bp}{\def\PY@tc##1{\textcolor[rgb]{0.00,0.50,0.00}{##1}}}
\@namedef{PY@tok@fm}{\def\PY@tc##1{\textcolor[rgb]{0.00,0.00,1.00}{##1}}}
\@namedef{PY@tok@vc}{\def\PY@tc##1{\textcolor[rgb]{0.10,0.09,0.49}{##1}}}
\@namedef{PY@tok@vg}{\def\PY@tc##1{\textcolor[rgb]{0.10,0.09,0.49}{##1}}}
\@namedef{PY@tok@vi}{\def\PY@tc##1{\textcolor[rgb]{0.10,0.09,0.49}{##1}}}
\@namedef{PY@tok@vm}{\def\PY@tc##1{\textcolor[rgb]{0.10,0.09,0.49}{##1}}}
\@namedef{PY@tok@sa}{\def\PY@tc##1{\textcolor[rgb]{0.73,0.13,0.13}{##1}}}
\@namedef{PY@tok@sb}{\def\PY@tc##1{\textcolor[rgb]{0.73,0.13,0.13}{##1}}}
\@namedef{PY@tok@sc}{\def\PY@tc##1{\textcolor[rgb]{0.73,0.13,0.13}{##1}}}
\@namedef{PY@tok@dl}{\def\PY@tc##1{\textcolor[rgb]{0.73,0.13,0.13}{##1}}}
\@namedef{PY@tok@s2}{\def\PY@tc##1{\textcolor[rgb]{0.73,0.13,0.13}{##1}}}
\@namedef{PY@tok@sh}{\def\PY@tc##1{\textcolor[rgb]{0.73,0.13,0.13}{##1}}}
\@namedef{PY@tok@s1}{\def\PY@tc##1{\textcolor[rgb]{0.73,0.13,0.13}{##1}}}
\@namedef{PY@tok@mb}{\def\PY@tc##1{\textcolor[rgb]{0.40,0.40,0.40}{##1}}}
\@namedef{PY@tok@mf}{\def\PY@tc##1{\textcolor[rgb]{0.40,0.40,0.40}{##1}}}
\@namedef{PY@tok@mh}{\def\PY@tc##1{\textcolor[rgb]{0.40,0.40,0.40}{##1}}}
\@namedef{PY@tok@mi}{\def\PY@tc##1{\textcolor[rgb]{0.40,0.40,0.40}{##1}}}
\@namedef{PY@tok@il}{\def\PY@tc##1{\textcolor[rgb]{0.40,0.40,0.40}{##1}}}
\@namedef{PY@tok@mo}{\def\PY@tc##1{\textcolor[rgb]{0.40,0.40,0.40}{##1}}}
\@namedef{PY@tok@ch}{\let\PY@it=\textit\def\PY@tc##1{\textcolor[rgb]{0.24,0.48,0.48}{##1}}}
\@namedef{PY@tok@cm}{\let\PY@it=\textit\def\PY@tc##1{\textcolor[rgb]{0.24,0.48,0.48}{##1}}}
\@namedef{PY@tok@cpf}{\let\PY@it=\textit\def\PY@tc##1{\textcolor[rgb]{0.24,0.48,0.48}{##1}}}
\@namedef{PY@tok@c1}{\let\PY@it=\textit\def\PY@tc##1{\textcolor[rgb]{0.24,0.48,0.48}{##1}}}
\@namedef{PY@tok@cs}{\let\PY@it=\textit\def\PY@tc##1{\textcolor[rgb]{0.24,0.48,0.48}{##1}}}

\def\PYZbs{\char`\\}
\def\PYZus{\char`\_}
\def\PYZob{\char`\{}
\def\PYZcb{\char`\}}
\def\PYZca{\char`\^}
\def\PYZam{\char`\&}
\def\PYZlt{\char`\<}
\def\PYZgt{\char`\>}
\def\PYZsh{\char`\#}
\def\PYZpc{\char`\%}
\def\PYZdl{\char`\$}
\def\PYZhy{\char`\-}
\def\PYZsq{\char`\'}
\def\PYZdq{\char`\"}
\def\PYZti{\char`\~}
% for compatibility with earlier versions
\def\PYZat{@}
\def\PYZlb{[}
\def\PYZrb{]}
\makeatother


    % For linebreaks inside Verbatim environment from package fancyvrb. 
    \makeatletter
        \newbox\Wrappedcontinuationbox 
        \newbox\Wrappedvisiblespacebox 
        \newcommand*\Wrappedvisiblespace {\textcolor{red}{\textvisiblespace}} 
        \newcommand*\Wrappedcontinuationsymbol {\textcolor{red}{\llap{\tiny$\m@th\hookrightarrow$}}} 
        \newcommand*\Wrappedcontinuationindent {3ex } 
        \newcommand*\Wrappedafterbreak {\kern\Wrappedcontinuationindent\copy\Wrappedcontinuationbox} 
        % Take advantage of the already applied Pygments mark-up to insert 
        % potential linebreaks for TeX processing. 
        %        {, <, #, %, $, ' and ": go to next line. 
        %        _, }, ^, &, >, - and ~: stay at end of broken line. 
        % Use of \textquotesingle for straight quote. 
        \newcommand*\Wrappedbreaksatspecials {% 
            \def\PYGZus{\discretionary{\char`\_}{\Wrappedafterbreak}{\char`\_}}% 
            \def\PYGZob{\discretionary{}{\Wrappedafterbreak\char`\{}{\char`\{}}% 
            \def\PYGZcb{\discretionary{\char`\}}{\Wrappedafterbreak}{\char`\}}}% 
            \def\PYGZca{\discretionary{\char`\^}{\Wrappedafterbreak}{\char`\^}}% 
            \def\PYGZam{\discretionary{\char`\&}{\Wrappedafterbreak}{\char`\&}}% 
            \def\PYGZlt{\discretionary{}{\Wrappedafterbreak\char`\<}{\char`\<}}% 
            \def\PYGZgt{\discretionary{\char`\>}{\Wrappedafterbreak}{\char`\>}}% 
            \def\PYGZsh{\discretionary{}{\Wrappedafterbreak\char`\#}{\char`\#}}% 
            \def\PYGZpc{\discretionary{}{\Wrappedafterbreak\char`\%}{\char`\%}}% 
            \def\PYGZdl{\discretionary{}{\Wrappedafterbreak\char`\$}{\char`\$}}% 
            \def\PYGZhy{\discretionary{\char`\-}{\Wrappedafterbreak}{\char`\-}}% 
            \def\PYGZsq{\discretionary{}{\Wrappedafterbreak\textquotesingle}{\textquotesingle}}% 
            \def\PYGZdq{\discretionary{}{\Wrappedafterbreak\char`\"}{\char`\"}}% 
            \def\PYGZti{\discretionary{\char`\~}{\Wrappedafterbreak}{\char`\~}}% 
        } 
        % Some characters . , ; ? ! / are not pygmentized. 
        % This macro makes them "active" and they will insert potential linebreaks 
        \newcommand*\Wrappedbreaksatpunct {% 
            \lccode`\~`\.\lowercase{\def~}{\discretionary{\hbox{\char`\.}}{\Wrappedafterbreak}{\hbox{\char`\.}}}% 
            \lccode`\~`\,\lowercase{\def~}{\discretionary{\hbox{\char`\,}}{\Wrappedafterbreak}{\hbox{\char`\,}}}% 
            \lccode`\~`\;\lowercase{\def~}{\discretionary{\hbox{\char`\;}}{\Wrappedafterbreak}{\hbox{\char`\;}}}% 
            \lccode`\~`\:\lowercase{\def~}{\discretionary{\hbox{\char`\:}}{\Wrappedafterbreak}{\hbox{\char`\:}}}% 
            \lccode`\~`\?\lowercase{\def~}{\discretionary{\hbox{\char`\?}}{\Wrappedafterbreak}{\hbox{\char`\?}}}% 
            \lccode`\~`\!\lowercase{\def~}{\discretionary{\hbox{\char`\!}}{\Wrappedafterbreak}{\hbox{\char`\!}}}% 
            \lccode`\~`\/\lowercase{\def~}{\discretionary{\hbox{\char`\/}}{\Wrappedafterbreak}{\hbox{\char`\/}}}% 
            \catcode`\.\active
            \catcode`\,\active 
            \catcode`\;\active
            \catcode`\:\active
            \catcode`\?\active
            \catcode`\!\active
            \catcode`\/\active 
            \lccode`\~`\~ 	
        }
    \makeatother

    \let\OriginalVerbatim=\Verbatim
    \makeatletter
    \renewcommand{\Verbatim}[1][1]{%
        %\parskip\z@skip
        \sbox\Wrappedcontinuationbox {\Wrappedcontinuationsymbol}%
        \sbox\Wrappedvisiblespacebox {\FV@SetupFont\Wrappedvisiblespace}%
        \def\FancyVerbFormatLine ##1{\hsize\linewidth
            \vtop{\raggedright\hyphenpenalty\z@\exhyphenpenalty\z@
                \doublehyphendemerits\z@\finalhyphendemerits\z@
                \strut ##1\strut}%
        }%
        % If the linebreak is at a space, the latter will be displayed as visible
        % space at end of first line, and a continuation symbol starts next line.
        % Stretch/shrink are however usually zero for typewriter font.
        \def\FV@Space {%
            \nobreak\hskip\z@ plus\fontdimen3\font minus\fontdimen4\font
            \discretionary{\copy\Wrappedvisiblespacebox}{\Wrappedafterbreak}
            {\kern\fontdimen2\font}%
        }%
        
        % Allow breaks at special characters using \PYG... macros.
        \Wrappedbreaksatspecials
        % Breaks at punctuation characters . , ; ? ! and / need catcode=\active 	
        \OriginalVerbatim[#1,codes*=\Wrappedbreaksatpunct]%
    }
    \makeatother

    % Exact colors from NB
    \definecolor{incolor}{HTML}{303F9F}
    \definecolor{outcolor}{HTML}{D84315}
    \definecolor{cellborder}{HTML}{CFCFCF}
    \definecolor{cellbackground}{HTML}{F7F7F7}
    
    % prompt
    \makeatletter
    \newcommand{\boxspacing}{\kern\kvtcb@left@rule\kern\kvtcb@boxsep}
    \makeatother
    \newcommand{\prompt}[4]{
        {\ttfamily\llap{{\color{#2}[#3]:\hspace{3pt}#4}}\vspace{-\baselineskip}}
    }
    

    
    % Prevent overflowing lines due to hard-to-break entities
    \sloppy 
    % Setup hyperref package
    \hypersetup{
      breaklinks=true,  % so long urls are correctly broken across lines
      colorlinks=true,
      urlcolor=urlcolor,
      linkcolor=linkcolor,
      citecolor=citecolor,
      }
    % Slightly bigger margins than the latex defaults
    
    \geometry{verbose,tmargin=1in,bmargin=1in,lmargin=1in,rmargin=1in}
    
    

\begin{document}
    
    \maketitle
    
    

    
    \begin{tcolorbox}[breakable, size=fbox, boxrule=1pt, pad at break*=1mm,colback=cellbackground, colframe=cellborder]
\prompt{In}{incolor}{287}{\boxspacing}
\begin{Verbatim}[commandchars=\\\{\}]
\PY{n+nf}{library}\PY{p}{(}\PY{n}{repr}\PY{p}{)}
\PY{n+nf}{options}\PY{p}{(}\PY{n}{repr.plot.width} \PY{o}{=} \PY{l+m}{12}\PY{p}{,} \PY{n}{repr.plot.height} \PY{o}{=} \PY{l+m}{12}\PY{p}{)}
\end{Verbatim}
\end{tcolorbox}

    \section{Exercise 1}\label{exercise-1}

\textbf{\emph{Read the file ldeaths in the folder datasets of R. Make
the graphical representation. Identify and estimate the trend, the
seasonal component and the residual component. Are the residuals a
sample of an IID noise?}}

    Firstly, we'll plot the data of the ldeaths dataset from the R datasets.

    \begin{tcolorbox}[breakable, size=fbox, boxrule=1pt, pad at break*=1mm,colback=cellbackground, colframe=cellborder]
\prompt{In}{incolor}{395}{\boxspacing}
\begin{Verbatim}[commandchars=\\\{\}]
\PY{n}{data\PYZus{}ldeaths} \PY{o}{\PYZlt{}\PYZhy{}} \PY{n}{ldeaths}

\PY{n+nf}{plot}\PY{p}{(}\PY{n}{data\PYZus{}ldeaths}\PY{p}{,}
     \PY{n}{main}\PY{o}{=}\PY{l+s}{\PYZdq{}}\PY{l+s}{Monthly Deaths from Lung Diseases in the UK (1974\PYZhy{}1979)\PYZdq{}}\PY{p}{,}
     \PY{n}{ylab}\PY{o}{=}\PY{l+s}{\PYZdq{}}\PY{l+s}{Number of Deaths\PYZdq{}}\PY{p}{,}
     \PY{n}{xlab}\PY{o}{=}\PY{l+s}{\PYZdq{}}\PY{l+s}{Year\PYZdq{}}\PY{p}{,}
     \PY{n}{col}\PY{o}{=}\PY{l+s}{\PYZdq{}}\PY{l+s}{blue\PYZdq{}}\PY{p}{,}
     \PY{n}{lwd}\PY{o}{=}\PY{l+m}{2}\PY{p}{,}
     \PY{n}{type}\PY{o}{=}\PY{l+s}{\PYZdq{}}\PY{l+s}{o\PYZdq{}}\PY{p}{,}
     \PY{n}{pch}\PY{o}{=}\PY{l+m}{20}\PY{p}{)}

\PY{n+nf}{grid}\PY{p}{(}\PY{p}{)}
\end{Verbatim}
\end{tcolorbox}

    \begin{center}
    \adjustimage{max size={0.9\linewidth}{0.9\paperheight}}{output_3_0.png}
    \end{center}
    { \hspace*{\fill} \\}
    
    \begin{tcolorbox}[breakable, size=fbox, boxrule=1pt, pad at break*=1mm,colback=cellbackground, colframe=cellborder]
\prompt{In}{incolor}{396}{\boxspacing}
\begin{Verbatim}[commandchars=\\\{\}]
\PY{n}{t} \PY{o}{\PYZlt{}\PYZhy{}} \PY{l+m}{1}\PY{o}{:}\PY{p}{(}\PY{l+m}{6}\PY{o}{*}\PY{l+m}{12}\PY{p}{)} \PY{c+c1}{\PYZsh{} Generate the indexes of the timepoints}
\end{Verbatim}
\end{tcolorbox}

    \begin{tcolorbox}[breakable, size=fbox, boxrule=1pt, pad at break*=1mm,colback=cellbackground, colframe=cellborder]
\prompt{In}{incolor}{397}{\boxspacing}
\begin{Verbatim}[commandchars=\\\{\}]
\PY{n+nf}{layout }\PY{p}{(}\PY{l+m}{1}\PY{o}{:}\PY{l+m}{2}\PY{p}{)}
\PY{n+nf}{plot}\PY{p}{(}\PY{n+nf}{aggregate}\PY{p}{(}\PY{n}{data\PYZus{}ldeaths}\PY{p}{)}\PY{p}{,} \PY{n}{main}\PY{o}{=}\PY{l+s}{\PYZdq{}}\PY{l+s}{Deaths across time by year\PYZdq{}}\PY{p}{)}
\PY{n+nf}{boxplot}\PY{p}{(}\PY{n}{data\PYZus{}ldeaths}\PY{o}{\PYZti{}}\PY{n+nf}{cycle}\PY{p}{(}\PY{n}{data\PYZus{}ldeaths}\PY{p}{)}\PY{p}{,} \PY{n}{main}\PY{o}{=}\PY{l+s}{\PYZdq{}}\PY{l+s}{Deaths across time by month\PYZdq{}}\PY{p}{)}
\end{Verbatim}
\end{tcolorbox}

    \begin{center}
    \adjustimage{max size={0.9\linewidth}{0.9\paperheight}}{output_5_0.png}
    \end{center}
    { \hspace*{\fill} \\}
    
    Here we can see the data across years, and the mean and variance of the
month cycles.

    \begin{tcolorbox}[breakable, size=fbox, boxrule=1pt, pad at break*=1mm,colback=cellbackground, colframe=cellborder]
\prompt{In}{incolor}{398}{\boxspacing}
\begin{Verbatim}[commandchars=\\\{\}]
\PY{n}{y}\PY{o}{\PYZlt{}\PYZhy{}}\PY{n+nf}{log}\PY{p}{(}\PY{n}{data\PYZus{}ldeaths}\PY{p}{)}
\PY{n+nf}{plot }\PY{p}{(}\PY{n}{t}\PY{p}{,} \PY{n}{y}\PY{p}{,} \PY{n}{xlab}\PY{o}{=}\PY{l+s}{\PYZdq{}}\PY{l+s}{time\PYZdq{}}\PY{p}{,} \PY{n}{ylab}\PY{o}{=}\PY{l+s}{\PYZdq{}}\PY{l+s}{deaths\PYZdq{}}\PY{p}{,} \PY{n}{type}\PY{o}{=}\PY{l+s}{\PYZdq{}}\PY{l+s}{o\PYZdq{}} \PY{p}{,} \PY{n}{main}\PY{o}{=}\PY{l+s}{\PYZdq{}}\PY{l+s}{Logarithmic Transformation of Lung Deaths\PYZdq{}}\PY{p}{)}
\end{Verbatim}
\end{tcolorbox}

    \begin{center}
    \adjustimage{max size={0.9\linewidth}{0.9\paperheight}}{output_7_0.png}
    \end{center}
    { \hspace*{\fill} \\}
    
    We now proceed to calculate the linear trend of the data, as we can see
we get an intercept \(b\) of approximately 7.7223 and a \(t\) of -0.0036
in: \[y=tX+b\]

    \begin{tcolorbox}[breakable, size=fbox, boxrule=1pt, pad at break*=1mm,colback=cellbackground, colframe=cellborder]
\prompt{In}{incolor}{399}{\boxspacing}
\begin{Verbatim}[commandchars=\\\{\}]
\PY{n}{line}\PY{o}{\PYZlt{}\PYZhy{}}\PY{n+nf}{lm}\PY{p}{(}\PY{n}{y}\PY{o}{\PYZti{}}\PY{n}{t}\PY{p}{)}
\PY{n+nf}{summary}\PY{p}{(}\PY{n}{line}\PY{p}{)}
\PY{n+nf}{plot}\PY{p}{(}\PY{n}{t}\PY{p}{,}\PY{n}{y}\PY{p}{,}\PY{n}{type}\PY{o}{=}\PY{l+s}{\PYZdq{}}\PY{l+s}{o\PYZdq{}}\PY{p}{,}\PY{n}{xlab}\PY{o}{=}\PY{l+s}{\PYZdq{}}\PY{l+s}{time\PYZdq{}}\PY{p}{,}\PY{n}{ylab}\PY{o}{=}\PY{l+s}{\PYZdq{}}\PY{l+s}{deaths\PYZdq{}}\PY{p}{,} \PY{n}{main}\PY{o}{=}\PY{l+s}{\PYZdq{}}\PY{l+s}{Linear trend of Lung Deaths\PYZdq{}}\PY{p}{)}\PY{p}{;}\PY{n+nf}{abline}\PY{p}{(}\PY{n}{line}\PY{p}{)}
\end{Verbatim}
\end{tcolorbox}

    
    \begin{Verbatim}[commandchars=\\\{\}]

Call:
lm(formula = y \textasciitilde{} t)

Residuals:
     Min       1Q   Median       3Q      Max 
-0.43429 -0.22570 -0.04839  0.20538  0.63989 

Coefficients:
             Estimate Std. Error t value Pr(>|t|)    
(Intercept)  7.722376   0.065789 117.381   <2e-16 ***
t           -0.003686   0.001566  -2.353   0.0214 *  
---
Signif. codes:  0 ‘***’ 0.001 ‘**’ 0.01 ‘*’ 0.05 ‘.’ 0.1 ‘ ’ 1

Residual standard error: 0.2762 on 70 degrees of freedom
Multiple R-squared:  0.07332,	Adjusted R-squared:  0.06009 
F-statistic: 5.539 on 1 and 70 DF,  p-value: 0.02141

    \end{Verbatim}

    
    \begin{center}
    \adjustimage{max size={0.9\linewidth}{0.9\paperheight}}{output_9_1.png}
    \end{center}
    { \hspace*{\fill} \\}
    
    Now we can remove the trend from the data.

    \begin{tcolorbox}[breakable, size=fbox, boxrule=1pt, pad at break*=1mm,colback=cellbackground, colframe=cellborder]
\prompt{In}{incolor}{400}{\boxspacing}
\begin{Verbatim}[commandchars=\\\{\}]
\PY{n}{y}\PY{o}{\PYZlt{}\PYZhy{}}\PY{n+nf}{log}\PY{p}{(}\PY{n}{data\PYZus{}ldeaths}\PY{p}{)}
\PY{n+nf}{lm}\PY{p}{(}\PY{n}{y}\PY{o}{\PYZti{}}\PY{n}{t}\PY{p}{)}
\PY{n}{trend}\PY{o}{\PYZlt{}\PYZhy{}} \PY{l+m}{\PYZhy{}0.003686}\PY{o}{*}\PY{n}{t}\PY{l+m}{+7.722376}
\PY{n}{z}\PY{o}{\PYZlt{}\PYZhy{}}\PY{n}{y}\PY{o}{\PYZhy{}}\PY{n}{trend}
\PY{n+nf}{plot.ts}\PY{p}{(}\PY{n}{z}\PY{p}{,} \PY{n}{main}\PY{o}{=}\PY{l+s}{\PYZdq{}}\PY{l+s}{Data without trend\PYZdq{}}\PY{p}{)}
\end{Verbatim}
\end{tcolorbox}

    
    \begin{Verbatim}[commandchars=\\\{\}]

Call:
lm(formula = y \textasciitilde{} t)

Coefficients:
(Intercept)            t  
   7.722376    -0.003686  

    \end{Verbatim}

    
    \begin{center}
    \adjustimage{max size={0.9\linewidth}{0.9\paperheight}}{output_11_1.png}
    \end{center}
    { \hspace*{\fill} \\}
    
    We can now compute the seasonality component by segmenting the time
series into a matrix with 12 columns that represent the months.

    \begin{tcolorbox}[breakable, size=fbox, boxrule=1pt, pad at break*=1mm,colback=cellbackground, colframe=cellborder]
\prompt{In}{incolor}{401}{\boxspacing}
\begin{Verbatim}[commandchars=\\\{\}]
\PY{n}{z} \PY{o}{\PYZlt{}\PYZhy{}} \PY{n+nf}{matrix}\PY{p}{(}\PY{n}{z}\PY{p}{,} \PY{n}{ncol}\PY{o}{=}\PY{l+m}{12}\PY{p}{,} \PY{n}{byrow}\PY{o}{=}\PY{k+kc}{TRUE}\PY{p}{)}
\PY{n}{z} \PY{o}{\PYZlt{}\PYZhy{}} \PY{n+nf}{t}\PY{p}{(}\PY{n}{z}\PY{p}{)}
\end{Verbatim}
\end{tcolorbox}

    \begin{tcolorbox}[breakable, size=fbox, boxrule=1pt, pad at break*=1mm,colback=cellbackground, colframe=cellborder]
\prompt{In}{incolor}{402}{\boxspacing}
\begin{Verbatim}[commandchars=\\\{\}]
\PY{n}{e}\PY{o}{\PYZlt{}\PYZhy{}}\PY{n+nf}{apply}\PY{p}{(}\PY{n}{z}\PY{p}{,}\PY{l+m}{1}\PY{p}{,}\PY{n}{mean}\PY{p}{)}
\PY{n}{ee}\PY{o}{\PYZlt{}\PYZhy{}}\PY{n+nf}{mean}\PY{p}{(}\PY{n}{e}\PY{p}{)}
\PY{n}{s}\PY{o}{\PYZlt{}\PYZhy{}}\PY{n}{e}\PY{o}{\PYZhy{}}\PY{n}{ee}
\PY{n+nf}{plot }\PY{p}{(}\PY{n}{s}\PY{p}{,}\PY{n}{type}\PY{o}{=}\PY{l+s}{\PYZdq{}}\PY{l+s}{o\PYZdq{}}\PY{p}{,} \PY{n}{main}\PY{o}{=}\PY{l+s}{\PYZdq{}}\PY{l+s}{Seasonality Component\PYZdq{}}\PY{p}{)}
\end{Verbatim}
\end{tcolorbox}

    \begin{center}
    \adjustimage{max size={0.9\linewidth}{0.9\paperheight}}{output_14_0.png}
    \end{center}
    { \hspace*{\fill} \\}
    
    We can now repeat this component to match the length of the original
series and substract it from the data.

    \begin{tcolorbox}[breakable, size=fbox, boxrule=1pt, pad at break*=1mm,colback=cellbackground, colframe=cellborder]
\prompt{In}{incolor}{403}{\boxspacing}
\begin{Verbatim}[commandchars=\\\{\}]
\PY{n}{est}\PY{o}{\PYZlt{}\PYZhy{}}\PY{n+nf}{array}\PY{p}{(}\PY{n}{s}\PY{p}{,}\PY{l+m}{12}\PY{o}{*}\PY{l+m}{6}\PY{p}{)}
\PY{n+nf}{plot }\PY{p}{(}\PY{n}{est}\PY{p}{,}\PY{n}{type}\PY{o}{=}\PY{l+s}{\PYZdq{}}\PY{l+s}{o\PYZdq{}}\PY{p}{,} \PY{n}{main}\PY{o}{=}\PY{l+s}{\PYZdq{}}\PY{l+s}{Seasonality component of length n\PYZdq{}}\PY{p}{)}
\end{Verbatim}
\end{tcolorbox}

    \begin{center}
    \adjustimage{max size={0.9\linewidth}{0.9\paperheight}}{output_16_0.png}
    \end{center}
    { \hspace*{\fill} \\}
    
    \begin{tcolorbox}[breakable, size=fbox, boxrule=1pt, pad at break*=1mm,colback=cellbackground, colframe=cellborder]
\prompt{In}{incolor}{404}{\boxspacing}
\begin{Verbatim}[commandchars=\\\{\}]
\PY{n+nf}{dim}\PY{p}{(}\PY{n}{z}\PY{p}{)}\PY{o}{\PYZlt{}\PYZhy{}}\PY{l+m}{12}\PY{o}{*}\PY{l+m}{6}
\PY{n}{u}\PY{o}{\PYZlt{}\PYZhy{}}\PY{n}{z}\PY{o}{\PYZhy{}}\PY{n}{est}
\PY{n+nf}{plot }\PY{p}{(}\PY{n}{u}\PY{p}{,}\PY{n}{type}\PY{o}{=}\PY{l+s}{\PYZdq{}}\PY{l+s}{o\PYZdq{}}\PY{p}{,} \PY{n}{main}\PY{o}{=}\PY{l+s}{\PYZdq{}}\PY{l+s}{Residuals of the data\PYZdq{}}\PY{p}{)}
\end{Verbatim}
\end{tcolorbox}

    \begin{center}
    \adjustimage{max size={0.9\linewidth}{0.9\paperheight}}{output_17_0.png}
    \end{center}
    { \hspace*{\fill} \\}
    
    This last plot represents the residuals of the data, let us now check if
it truly is IID noise.

    \begin{tcolorbox}[breakable, size=fbox, boxrule=1pt, pad at break*=1mm,colback=cellbackground, colframe=cellborder]
\prompt{In}{incolor}{405}{\boxspacing}
\begin{Verbatim}[commandchars=\\\{\}]
\PY{n}{n}\PY{o}{\PYZlt{}\PYZhy{}}\PY{l+m}{1000}
\PY{n}{z1}\PY{o}{\PYZlt{}\PYZhy{}}\PY{n+nf}{rnorm}\PY{p}{(}\PY{n}{n}\PY{p}{)}
\PY{n+nf}{plot}\PY{p}{(}\PY{n}{z1}\PY{p}{,}\PY{n}{type}\PY{o}{=}\PY{l+s}{\PYZdq{}}\PY{l+s}{l\PYZdq{}}\PY{p}{,} \PY{n}{main}\PY{o}{=}\PY{l+s}{\PYZdq{}}\PY{l+s}{Random Gaussian Noise\PYZdq{}}\PY{p}{)}
\end{Verbatim}
\end{tcolorbox}

    \begin{center}
    \adjustimage{max size={0.9\linewidth}{0.9\paperheight}}{output_19_0.png}
    \end{center}
    { \hspace*{\fill} \\}
    
    \begin{tcolorbox}[breakable, size=fbox, boxrule=1pt, pad at break*=1mm,colback=cellbackground, colframe=cellborder]
\prompt{In}{incolor}{406}{\boxspacing}
\begin{Verbatim}[commandchars=\\\{\}]
\PY{n+nf}{par}\PY{p}{(}\PY{n}{mfrow} \PY{o}{=} \PY{n+nf}{c}\PY{p}{(}\PY{l+m}{2}\PY{p}{,} \PY{l+m}{2}\PY{p}{)}\PY{p}{)}
\PY{n+nf}{hist}\PY{p}{(}\PY{n}{z1}\PY{p}{,}\PY{n}{freq}\PY{o}{=}\PY{n+nb+bp}{F}\PY{p}{,} \PY{n}{main} \PY{o}{=} \PY{l+s}{\PYZdq{}}\PY{l+s}{Noise Histogram\PYZdq{}}\PY{p}{)}
\PY{n+nf}{hist}\PY{p}{(}\PY{n}{u}\PY{p}{,}\PY{n}{freq}\PY{o}{=}\PY{n+nb+bp}{F}\PY{p}{,} \PY{n}{main} \PY{o}{=} \PY{l+s}{\PYZdq{}}\PY{l+s}{Residuals Histogram\PYZdq{}}\PY{p}{)}
\PY{n+nf}{acf}\PY{p}{(}\PY{n}{z1}\PY{p}{,} \PY{n}{main} \PY{o}{=} \PY{l+s}{\PYZdq{}}\PY{l+s}{Noise Correlogram\PYZdq{}}\PY{p}{)}
\PY{n+nf}{acf}\PY{p}{(}\PY{n}{u}\PY{p}{,} \PY{n}{main} \PY{o}{=} \PY{l+s}{\PYZdq{}}\PY{l+s}{Residuals Correlogram\PYZdq{}}\PY{p}{)}
\PY{n+nf}{par}\PY{p}{(}\PY{n}{mfrow} \PY{o}{=} \PY{n+nf}{c}\PY{p}{(}\PY{l+m}{1}\PY{p}{,} \PY{l+m}{1}\PY{p}{)}\PY{p}{)}
\end{Verbatim}
\end{tcolorbox}

    \begin{center}
    \adjustimage{max size={0.9\linewidth}{0.9\paperheight}}{output_20_0.png}
    \end{center}
    { \hspace*{\fill} \\}
    
    \begin{tcolorbox}[breakable, size=fbox, boxrule=1pt, pad at break*=1mm,colback=cellbackground, colframe=cellborder]
\prompt{In}{incolor}{407}{\boxspacing}
\begin{Verbatim}[commandchars=\\\{\}]
\PY{n+nf}{Box.test }\PY{p}{(}\PY{n}{z1}\PY{p}{,} \PY{n}{lag}\PY{o}{=}\PY{l+m}{1}\PY{p}{,} \PY{n}{type}\PY{o}{=}\PY{n+nf}{c}\PY{p}{(}\PY{l+s}{\PYZdq{}}\PY{l+s}{Ljung\PYZhy{}Box\PYZdq{}}\PY{p}{)}\PY{p}{)}
\PY{n+nf}{Box.test }\PY{p}{(}\PY{n}{u}\PY{p}{,} \PY{n}{lag}\PY{o}{=}\PY{l+m}{1}\PY{p}{,} \PY{n}{type}\PY{o}{=}\PY{n+nf}{c}\PY{p}{(}\PY{l+s}{\PYZdq{}}\PY{l+s}{Ljung\PYZhy{}Box\PYZdq{}}\PY{p}{)}\PY{p}{)}
\end{Verbatim}
\end{tcolorbox}

    
    \begin{Verbatim}[commandchars=\\\{\}]

	Box-Ljung test

data:  z1
X-squared = 5.9906, df = 1, p-value = 0.01438

    \end{Verbatim}

    
    
    \begin{Verbatim}[commandchars=\\\{\}]

	Box-Ljung test

data:  u
X-squared = 4.7035, df = 1, p-value = 0.0301

    \end{Verbatim}

    
    \begin{tcolorbox}[breakable, size=fbox, boxrule=1pt, pad at break*=1mm,colback=cellbackground, colframe=cellborder]
\prompt{In}{incolor}{408}{\boxspacing}
\begin{Verbatim}[commandchars=\\\{\}]
\PY{n+nf}{par}\PY{p}{(}\PY{n}{mfrow} \PY{o}{=} \PY{n+nf}{c}\PY{p}{(}\PY{l+m}{2}\PY{p}{,}\PY{l+m}{1}\PY{p}{)}\PY{p}{)}
\PY{n+nf}{qqnorm}\PY{p}{(}\PY{n}{z1}\PY{p}{,} \PY{n}{main}\PY{o}{=}\PY{l+s}{\PYZdq{}}\PY{l+s}{Noise Q\PYZhy{}Q norm plot\PYZdq{}}\PY{p}{)}
\PY{n+nf}{qqline}\PY{p}{(}\PY{n}{z1}\PY{p}{,} \PY{n}{main}\PY{o}{=}\PY{l+s}{\PYZdq{}}\PY{l+s}{Noise Q\PYZhy{}Q line plot\PYZdq{}}\PY{p}{)}
\PY{n+nf}{qqnorm}\PY{p}{(}\PY{n}{u}\PY{p}{,} \PY{n}{main}\PY{o}{=}\PY{l+s}{\PYZdq{}}\PY{l+s}{Residuals Q\PYZhy{}Q norm plot\PYZdq{}}\PY{p}{)}
\PY{n+nf}{qqline}\PY{p}{(}\PY{n}{u}\PY{p}{,} \PY{n}{main}\PY{o}{=}\PY{l+s}{\PYZdq{}}\PY{l+s}{Residuals Q\PYZhy{}Q line plot\PYZdq{}}\PY{p}{)}
\PY{n+nf}{par}\PY{p}{(}\PY{n}{mfrow} \PY{o}{=} \PY{n+nf}{c}\PY{p}{(}\PY{l+m}{1}\PY{p}{,}\PY{l+m}{1}\PY{p}{)}\PY{p}{)}
\end{Verbatim}
\end{tcolorbox}

    \begin{center}
    \adjustimage{max size={0.9\linewidth}{0.9\paperheight}}{output_22_0.png}
    \end{center}
    { \hspace*{\fill} \\}
    
    \begin{tcolorbox}[breakable, size=fbox, boxrule=1pt, pad at break*=1mm,colback=cellbackground, colframe=cellborder]
\prompt{In}{incolor}{409}{\boxspacing}
\begin{Verbatim}[commandchars=\\\{\}]
\PY{n+nf}{shapiro.test}\PY{p}{(}\PY{n}{z1}\PY{p}{)}
\PY{n+nf}{shapiro.test}\PY{p}{(}\PY{n}{u}\PY{p}{)}
\end{Verbatim}
\end{tcolorbox}

    
    \begin{Verbatim}[commandchars=\\\{\}]

	Shapiro-Wilk normality test

data:  z1
W = 0.99891, p-value = 0.8259

    \end{Verbatim}

    
    
    \begin{Verbatim}[commandchars=\\\{\}]

	Shapiro-Wilk normality test

data:  u
W = 0.98442, p-value = 0.5167

    \end{Verbatim}

    
    As we can see, the correlogram and the tests seem to indicate that the
residuals are IID noise. Nevertheless, the residuals still have some
autocorrelation after \(h\) lags.

    \#Exercise 2 \textbf{\emph{Simulate a Gaussian white noise of n = 10.000
data. Verify by testing that it is an IID noise and a Gaussian white
noise. Simulate a Gaussian Random Walk. Simulate IID noises of 10.000
data that are not a Gaussian white noise: a Poisson noise and an
exponential noise. Test all what you can.}}

    Firstly let us simulate gaussian white noise with \(n=10000\).

    \begin{tcolorbox}[breakable, size=fbox, boxrule=1pt, pad at break*=1mm,colback=cellbackground, colframe=cellborder]
\prompt{In}{incolor}{428}{\boxspacing}
\begin{Verbatim}[commandchars=\\\{\}]
\PY{n}{n}\PY{o}{\PYZlt{}\PYZhy{}}\PY{l+m}{10000}
\PY{n}{white\PYZus{}noise\PYZus{}data}\PY{o}{\PYZlt{}\PYZhy{}}\PY{n+nf}{rnorm}\PY{p}{(}\PY{n}{n}\PY{p}{)} \PY{c+c1}{\PYZsh{} Generate the white noise}
\end{Verbatim}
\end{tcolorbox}

    \begin{tcolorbox}[breakable, size=fbox, boxrule=1pt, pad at break*=1mm,colback=cellbackground, colframe=cellborder]
\prompt{In}{incolor}{429}{\boxspacing}
\begin{Verbatim}[commandchars=\\\{\}]
\PY{n+nf}{par}\PY{p}{(}\PY{n}{mfrow} \PY{o}{=} \PY{n+nf}{c}\PY{p}{(}\PY{l+m}{3}\PY{p}{,} \PY{l+m}{1}\PY{p}{)}\PY{p}{)}
\PY{n+nf}{plot}\PY{p}{(}\PY{n}{white\PYZus{}noise\PYZus{}data}\PY{p}{,}\PY{n}{type}\PY{o}{=}\PY{l+s}{\PYZdq{}}\PY{l+s}{l\PYZdq{}}\PY{p}{,} \PY{n}{main}\PY{o}{=}\PY{l+s}{\PYZdq{}}\PY{l+s}{White Noise Data\PYZdq{}}\PY{p}{)}
\PY{n+nf}{hist}\PY{p}{(}\PY{n}{white\PYZus{}noise\PYZus{}data}\PY{p}{,}\PY{n}{freq}\PY{o}{=}\PY{n+nb+bp}{F}\PY{p}{,} \PY{n}{main}\PY{o}{=}\PY{l+s}{\PYZdq{}}\PY{l+s}{White Noise Histogram\PYZdq{}}\PY{p}{)}
\PY{n+nf}{acf}\PY{p}{(}\PY{n}{white\PYZus{}noise\PYZus{}data}\PY{p}{,} \PY{n}{main}\PY{o}{=}\PY{l+s}{\PYZdq{}}\PY{l+s}{White Noise Correlogram\PYZdq{}}\PY{p}{)}
\PY{n+nf}{par}\PY{p}{(}\PY{n}{mfrow} \PY{o}{=} \PY{n+nf}{c}\PY{p}{(}\PY{l+m}{1}\PY{p}{,} \PY{l+m}{1}\PY{p}{)}\PY{p}{)}
\end{Verbatim}
\end{tcolorbox}

    \begin{center}
    \adjustimage{max size={0.9\linewidth}{0.9\paperheight}}{output_28_0.png}
    \end{center}
    { \hspace*{\fill} \\}
    
    As we can see, this is the typical correlogram of an IID-noise.

    \begin{tcolorbox}[breakable, size=fbox, boxrule=1pt, pad at break*=1mm,colback=cellbackground, colframe=cellborder]
\prompt{In}{incolor}{431}{\boxspacing}
\begin{Verbatim}[commandchars=\\\{\}]
\PY{n}{gaussian\PYZus{}random\PYZus{}walk}\PY{o}{\PYZlt{}\PYZhy{}}\PY{n+nf}{cumsum}\PY{p}{(}\PY{n}{white\PYZus{}noise\PYZus{}data}\PY{p}{)}
\PY{n+nf}{plot}\PY{p}{(}\PY{n}{gaussian\PYZus{}random\PYZus{}walk}\PY{p}{,}\PY{n}{type}\PY{o}{=}\PY{l+s}{\PYZdq{}}\PY{l+s}{l\PYZdq{}}\PY{p}{,} \PY{n}{main}\PY{o}{=}\PY{l+s}{\PYZdq{}}\PY{l+s}{Random walk using the noise\PYZdq{}}\PY{p}{)}
\end{Verbatim}
\end{tcolorbox}

    \begin{center}
    \adjustimage{max size={0.9\linewidth}{0.9\paperheight}}{output_30_0.png}
    \end{center}
    { \hspace*{\fill} \\}
    
    \begin{tcolorbox}[breakable, size=fbox, boxrule=1pt, pad at break*=1mm,colback=cellbackground, colframe=cellborder]
\prompt{In}{incolor}{433}{\boxspacing}
\begin{Verbatim}[commandchars=\\\{\}]
\PY{n+nf}{acf}\PY{p}{(}\PY{n+nf}{diff}\PY{p}{(}\PY{n}{gaussian\PYZus{}random\PYZus{}walk}\PY{p}{)}\PY{p}{,} \PY{n}{main}\PY{o}{=}\PY{l+s}{\PYZdq{}}\PY{l+s}{The differences are still noise\PYZdq{}}\PY{p}{)}
\end{Verbatim}
\end{tcolorbox}

    \begin{center}
    \adjustimage{max size={0.9\linewidth}{0.9\paperheight}}{output_31_0.png}
    \end{center}
    { \hspace*{\fill} \\}
    
    Now we can test noises of non-gaussian sources: Poisson distribution and
exponentials.

    \begin{tcolorbox}[breakable, size=fbox, boxrule=1pt, pad at break*=1mm,colback=cellbackground, colframe=cellborder]
\prompt{In}{incolor}{307}{\boxspacing}
\begin{Verbatim}[commandchars=\\\{\}]
\PY{n}{n} \PY{o}{\PYZlt{}\PYZhy{}} \PY{l+m}{10000}
\PY{n}{poisson\PYZus{}noise} \PY{o}{\PYZlt{}\PYZhy{}} \PY{n+nf}{rpois }\PY{p}{(}\PY{n}{n}\PY{p}{,} \PY{n}{lambda}\PY{o}{=}\PY{l+m}{5}\PY{p}{)}
\PY{n}{exponential\PYZus{}noise} \PY{o}{\PYZlt{}\PYZhy{}} \PY{n+nf}{rexp}\PY{p}{(}\PY{n}{n}\PY{p}{,} \PY{n}{rate}\PY{o}{=}\PY{l+m}{10}\PY{p}{)}
\end{Verbatim}
\end{tcolorbox}

    \begin{tcolorbox}[breakable, size=fbox, boxrule=1pt, pad at break*=1mm,colback=cellbackground, colframe=cellborder]
\prompt{In}{incolor}{308}{\boxspacing}
\begin{Verbatim}[commandchars=\\\{\}]
\PY{n+nf}{par}\PY{p}{(}\PY{n}{mfrow} \PY{o}{=} \PY{n+nf}{c}\PY{p}{(}\PY{l+m}{3}\PY{p}{,} \PY{l+m}{1}\PY{p}{)}\PY{p}{)}
\PY{n+nf}{plot}\PY{p}{(}\PY{n}{poisson\PYZus{}noise}\PY{p}{,}\PY{n}{type}\PY{o}{=}\PY{l+s}{\PYZdq{}}\PY{l+s}{l\PYZdq{}}\PY{p}{,} \PY{n}{main}\PY{o}{=}\PY{l+s}{\PYZdq{}}\PY{l+s}{Poisson Noise Data\PYZdq{}}\PY{p}{)}
\PY{n+nf}{hist}\PY{p}{(}\PY{n}{poisson\PYZus{}noise}\PY{p}{,}\PY{n}{freq}\PY{o}{=}\PY{n+nb+bp}{F}\PY{p}{,} \PY{n}{main}\PY{o}{=}\PY{l+s}{\PYZdq{}}\PY{l+s}{Poisson Noise Histogram\PYZdq{}}\PY{p}{)}
\PY{n+nf}{acf}\PY{p}{(}\PY{n}{poisson\PYZus{}noise}\PY{p}{,} \PY{n}{main}\PY{o}{=}\PY{l+s}{\PYZdq{}}\PY{l+s}{Poisson Noise Correlogram\PYZdq{}}\PY{p}{)}
\PY{n+nf}{par}\PY{p}{(}\PY{n}{mfrow} \PY{o}{=} \PY{n+nf}{c}\PY{p}{(}\PY{l+m}{1}\PY{p}{,} \PY{l+m}{1}\PY{p}{)}\PY{p}{)}
\end{Verbatim}
\end{tcolorbox}

    \begin{center}
    \adjustimage{max size={0.9\linewidth}{0.9\paperheight}}{output_34_0.png}
    \end{center}
    { \hspace*{\fill} \\}
    
    \begin{tcolorbox}[breakable, size=fbox, boxrule=1pt, pad at break*=1mm,colback=cellbackground, colframe=cellborder]
\prompt{In}{incolor}{309}{\boxspacing}
\begin{Verbatim}[commandchars=\\\{\}]
\PY{n+nf}{par}\PY{p}{(}\PY{n}{mfrow} \PY{o}{=} \PY{n+nf}{c}\PY{p}{(}\PY{l+m}{3}\PY{p}{,} \PY{l+m}{1}\PY{p}{)}\PY{p}{)}
\PY{n+nf}{plot}\PY{p}{(}\PY{n}{exponential\PYZus{}noise}\PY{p}{,}\PY{n}{type}\PY{o}{=}\PY{l+s}{\PYZdq{}}\PY{l+s}{l\PYZdq{}}\PY{p}{,} \PY{n}{main}\PY{o}{=}\PY{l+s}{\PYZdq{}}\PY{l+s}{Exponential Noise Data\PYZdq{}}\PY{p}{)}
\PY{n+nf}{hist}\PY{p}{(}\PY{n}{exponential\PYZus{}noise}\PY{p}{,}\PY{n}{freq}\PY{o}{=}\PY{n+nb+bp}{F}\PY{p}{,} \PY{n}{main}\PY{o}{=}\PY{l+s}{\PYZdq{}}\PY{l+s}{Exponential Noise Histogram\PYZdq{}}\PY{p}{)}
\PY{n+nf}{acf}\PY{p}{(}\PY{n}{exponential\PYZus{}noise}\PY{p}{,} \PY{n}{main}\PY{o}{=}\PY{l+s}{\PYZdq{}}\PY{l+s}{Exponential Noise Correlogram\PYZdq{}}\PY{p}{)}
\PY{n+nf}{par}\PY{p}{(}\PY{n}{mfrow} \PY{o}{=} \PY{n+nf}{c}\PY{p}{(}\PY{l+m}{1}\PY{p}{,} \PY{l+m}{1}\PY{p}{)}\PY{p}{)}
\end{Verbatim}
\end{tcolorbox}

    \begin{center}
    \adjustimage{max size={0.9\linewidth}{0.9\paperheight}}{output_35_0.png}
    \end{center}
    { \hspace*{\fill} \\}
    
    \begin{tcolorbox}[breakable, size=fbox, boxrule=1pt, pad at break*=1mm,colback=cellbackground, colframe=cellborder]
\prompt{In}{incolor}{310}{\boxspacing}
\begin{Verbatim}[commandchars=\\\{\}]
\PY{n+nf}{Box.test }\PY{p}{(}\PY{n}{poisson\PYZus{}noise}\PY{p}{,} \PY{n}{lag}\PY{o}{=}\PY{l+m}{1}\PY{p}{,} \PY{n}{type}\PY{o}{=}\PY{n+nf}{c}\PY{p}{(}\PY{l+s}{\PYZdq{}}\PY{l+s}{Ljung\PYZhy{}Box\PYZdq{}}\PY{p}{)}\PY{p}{)}
\PY{n+nf}{Box.test }\PY{p}{(}\PY{n}{exponential\PYZus{}noise}\PY{p}{,} \PY{n}{lag}\PY{o}{=}\PY{l+m}{1}\PY{p}{,} \PY{n}{type}\PY{o}{=}\PY{n+nf}{c}\PY{p}{(}\PY{l+s}{\PYZdq{}}\PY{l+s}{Ljung\PYZhy{}Box\PYZdq{}}\PY{p}{)}\PY{p}{)}
\end{Verbatim}
\end{tcolorbox}

    
    \begin{Verbatim}[commandchars=\\\{\}]

	Box-Ljung test

data:  poisson\_noise
X-squared = 0.33487, df = 1, p-value = 0.5628

    \end{Verbatim}

    
    
    \begin{Verbatim}[commandchars=\\\{\}]

	Box-Ljung test

data:  exponential\_noise
X-squared = 0.12587, df = 1, p-value = 0.7228

    \end{Verbatim}

    
    \begin{tcolorbox}[breakable, size=fbox, boxrule=1pt, pad at break*=1mm,colback=cellbackground, colframe=cellborder]
\prompt{In}{incolor}{311}{\boxspacing}
\begin{Verbatim}[commandchars=\\\{\}]
\PY{n+nf}{par}\PY{p}{(}\PY{n}{mfrow} \PY{o}{=} \PY{n+nf}{c}\PY{p}{(}\PY{l+m}{2}\PY{p}{,}\PY{l+m}{1}\PY{p}{)}\PY{p}{)}
\PY{n+nf}{qqnorm}\PY{p}{(}\PY{n}{poisson\PYZus{}noise}\PY{p}{,} \PY{n}{main}\PY{o}{=}\PY{l+s}{\PYZdq{}}\PY{l+s}{Poisson Noise Q\PYZhy{}Q norm plot\PYZdq{}}\PY{p}{)}
\PY{n+nf}{qqline}\PY{p}{(}\PY{n}{poisson\PYZus{}noise}\PY{p}{)}
\PY{n+nf}{qqnorm}\PY{p}{(}\PY{n}{exponential\PYZus{}noise}\PY{p}{,} \PY{n}{main}\PY{o}{=}\PY{l+s}{\PYZdq{}}\PY{l+s}{Exponential Noise Q\PYZhy{}Q norm plot\PYZdq{}}\PY{p}{)}
\PY{n+nf}{qqline}\PY{p}{(}\PY{n}{exponential\PYZus{}noise}\PY{p}{)}
\PY{n+nf}{par}\PY{p}{(}\PY{n}{mfrow} \PY{o}{=} \PY{n+nf}{c}\PY{p}{(}\PY{l+m}{1}\PY{p}{,}\PY{l+m}{1}\PY{p}{)}\PY{p}{)}
\end{Verbatim}
\end{tcolorbox}

    \begin{center}
    \adjustimage{max size={0.9\linewidth}{0.9\paperheight}}{output_37_0.png}
    \end{center}
    { \hspace*{\fill} \\}
    
    \begin{tcolorbox}[breakable, size=fbox, boxrule=1pt, pad at break*=1mm,colback=cellbackground, colframe=cellborder]
\prompt{In}{incolor}{312}{\boxspacing}
\begin{Verbatim}[commandchars=\\\{\}]
\PY{n+nf}{shapiro.test}\PY{p}{(}\PY{n}{poisson\PYZus{}noise}\PY{p}{[}\PY{l+m}{1}\PY{o}{:}\PY{l+m}{5000}\PY{p}{]}\PY{p}{)}
\PY{n+nf}{shapiro.test}\PY{p}{(}\PY{n}{exponential\PYZus{}noise}\PY{p}{[}\PY{l+m}{1}\PY{o}{:}\PY{l+m}{5000}\PY{p}{]}\PY{p}{)}
\end{Verbatim}
\end{tcolorbox}

    
    \begin{Verbatim}[commandchars=\\\{\}]

	Shapiro-Wilk normality test

data:  poisson\_noise[1:5000]
W = 0.97202, p-value < 2.2e-16

    \end{Verbatim}

    
    
    \begin{Verbatim}[commandchars=\\\{\}]

	Shapiro-Wilk normality test

data:  exponential\_noise[1:5000]
W = 0.81829, p-value < 2.2e-16

    \end{Verbatim}

    
    As we can see, both the correlograms show that both of these samples are
noise. Nonetheless, using the Q-Q test and the shapiro test we can see
that they are not gaussian noise.

    \#Exercise 3 \textbf{\emph{Simulate an AR(p) model with 10000 data, for
p=1 and p=2. Fit the best model to the data in both cases. Validate the
model by showing the residuals are an IID noise.}}

    Firstly we will simulate an AR(1) model with \(n=10000\) with parameter
equal to 0.8792.

    \begin{tcolorbox}[breakable, size=fbox, boxrule=1pt, pad at break*=1mm,colback=cellbackground, colframe=cellborder]
\prompt{In}{incolor}{435}{\boxspacing}
\begin{Verbatim}[commandchars=\\\{\}]
\PY{n}{n}\PY{o}{\PYZlt{}\PYZhy{}}\PY{l+m}{10000}
\PY{n}{x\PYZus{}p1}\PY{o}{\PYZlt{}\PYZhy{}}\PY{n+nf}{arima.sim}\PY{p}{(}\PY{n+nf}{list}\PY{p}{(}\PY{n}{ar} \PY{o}{=} \PY{n+nf}{c}\PY{p}{(}\PY{l+m}{0.8792}\PY{p}{)}\PY{p}{,} \PY{n}{sd} \PY{o}{=} \PY{n+nf}{sqrt}\PY{p}{(}\PY{l+m}{0.1796}\PY{p}{)}\PY{p}{)}\PY{p}{,} \PY{n}{n}\PY{p}{)}
\PY{n+nf}{plot}\PY{p}{(}\PY{n}{x\PYZus{}p1}\PY{p}{,}\PY{n}{type}\PY{o}{=}\PY{l+s}{\PYZdq{}}\PY{l+s}{l\PYZdq{}}\PY{p}{,} \PY{n}{main}\PY{o}{=}\PY{l+s}{\PYZdq{}}\PY{l+s}{AR(1) simulation\PYZdq{}}\PY{p}{)}
\end{Verbatim}
\end{tcolorbox}

    \begin{center}
    \adjustimage{max size={0.9\linewidth}{0.9\paperheight}}{output_42_0.png}
    \end{center}
    { \hspace*{\fill} \\}
    
    Secondly, we will simulate an AR(2) model with \(n=10000\) with
parameters equal to \((0.8792,-0.1845)\).

    \begin{tcolorbox}[breakable, size=fbox, boxrule=1pt, pad at break*=1mm,colback=cellbackground, colframe=cellborder]
\prompt{In}{incolor}{437}{\boxspacing}
\begin{Verbatim}[commandchars=\\\{\}]
\PY{n}{x\PYZus{}p2}\PY{o}{\PYZlt{}\PYZhy{}}\PY{n+nf}{arima.sim}\PY{p}{(}\PY{n+nf}{list}\PY{p}{(}\PY{n}{ar} \PY{o}{=} \PY{n+nf}{c}\PY{p}{(}\PY{l+m}{0.8792}\PY{p}{,} \PY{l+m}{\PYZhy{}0.1845}\PY{p}{)}\PY{p}{,} \PY{n}{sd} \PY{o}{=} \PY{n+nf}{sqrt}\PY{p}{(}\PY{l+m}{0.1796}\PY{p}{)}\PY{p}{)}\PY{p}{,} \PY{n}{n}\PY{p}{)}
\PY{n+nf}{plot}\PY{p}{(}\PY{n}{x\PYZus{}p2}\PY{p}{,}\PY{n}{type}\PY{o}{=}\PY{l+s}{\PYZdq{}}\PY{l+s}{l\PYZdq{}}\PY{p}{,} \PY{n}{main}\PY{o}{=}\PY{l+s}{\PYZdq{}}\PY{l+s}{AR(2) simulation\PYZdq{}}\PY{p}{)}
\end{Verbatim}
\end{tcolorbox}

    \begin{center}
    \adjustimage{max size={0.9\linewidth}{0.9\paperheight}}{output_44_0.png}
    \end{center}
    { \hspace*{\fill} \\}
    
    We can now fit AR models to both simulations.

    \begin{tcolorbox}[breakable, size=fbox, boxrule=1pt, pad at break*=1mm,colback=cellbackground, colframe=cellborder]
\prompt{In}{incolor}{438}{\boxspacing}
\begin{Verbatim}[commandchars=\\\{\}]
\PY{n}{best\PYZus{}s1}\PY{o}{\PYZlt{}\PYZhy{}}\PY{n+nf}{ar}\PY{p}{(}\PY{n}{x\PYZus{}p1}\PY{p}{,} \PY{n}{order.max}\PY{o}{=}\PY{l+m}{3}\PY{p}{,} \PY{n}{method}\PY{o}{=}\PY{l+s}{\PYZdq{}}\PY{l+s}{mle\PYZdq{}}\PY{p}{)}
\PY{n}{best\PYZus{}s1}
\end{Verbatim}
\end{tcolorbox}

    
    \begin{Verbatim}[commandchars=\\\{\}]

Call:
ar(x = x\_p1, order.max = 3, method = "mle")

Coefficients:
     1  
0.8758  

Order selected 1  sigma\^{}2 estimated as  0.9889
    \end{Verbatim}

    
    \begin{tcolorbox}[breakable, size=fbox, boxrule=1pt, pad at break*=1mm,colback=cellbackground, colframe=cellborder]
\prompt{In}{incolor}{316}{\boxspacing}
\begin{Verbatim}[commandchars=\\\{\}]
\PY{n}{best\PYZus{}s2}\PY{o}{\PYZlt{}\PYZhy{}}\PY{n+nf}{ar}\PY{p}{(}\PY{n}{x\PYZus{}p2}\PY{p}{,} \PY{n}{order.max}\PY{o}{=}\PY{l+m}{3}\PY{p}{,} \PY{n}{method}\PY{o}{=}\PY{l+s}{\PYZdq{}}\PY{l+s}{mle\PYZdq{}}\PY{p}{)}
\PY{n}{best\PYZus{}s2}
\end{Verbatim}
\end{tcolorbox}

    
    \begin{Verbatim}[commandchars=\\\{\}]

Call:
ar(x = x\_p2, order.max = 3, method = "mle")

Coefficients:
      1        2  
 0.8237  -0.1905  

Order selected 2  sigma\^{}2 estimated as  0.9996
    \end{Verbatim}

    
    \begin{tcolorbox}[breakable, size=fbox, boxrule=1pt, pad at break*=1mm,colback=cellbackground, colframe=cellborder]
\prompt{In}{incolor}{317}{\boxspacing}
\begin{Verbatim}[commandchars=\\\{\}]
\PY{n}{s\PYZus{}p1}\PY{o}{\PYZlt{}\PYZhy{}}\PY{n+nf}{arima}\PY{p}{(}\PY{n}{x\PYZus{}p1}\PY{p}{,} \PY{n}{order}\PY{o}{=}\PY{n+nf}{c}\PY{p}{(}\PY{l+m}{1}\PY{p}{,}\PY{l+m}{0}\PY{p}{,}\PY{l+m}{0}\PY{p}{)}\PY{p}{,} \PY{n}{method}\PY{o}{=}\PY{l+s}{\PYZdq{}}\PY{l+s}{ML\PYZdq{}}\PY{p}{)}
\PY{n}{s\PYZus{}p1}
\end{Verbatim}
\end{tcolorbox}

    
    \begin{Verbatim}[commandchars=\\\{\}]

Call:
arima(x = x\_p1, order = c(1, 0, 0), method = "ML")

Coefficients:
         ar1  intercept
      0.8552    -0.5755
s.e.  0.0166     0.2168

sigma\^{}2 estimated as 0.9969:  log likelihood = -1418.05,  aic = 2842.1
    \end{Verbatim}

    
    \begin{tcolorbox}[breakable, size=fbox, boxrule=1pt, pad at break*=1mm,colback=cellbackground, colframe=cellborder]
\prompt{In}{incolor}{318}{\boxspacing}
\begin{Verbatim}[commandchars=\\\{\}]
\PY{n}{s\PYZus{}p2}\PY{o}{\PYZlt{}\PYZhy{}}\PY{n+nf}{arima}\PY{p}{(}\PY{n}{x\PYZus{}p2}\PY{p}{,} \PY{n}{order}\PY{o}{=}\PY{n+nf}{c}\PY{p}{(}\PY{l+m}{2}\PY{p}{,}\PY{l+m}{0}\PY{p}{,}\PY{l+m}{0}\PY{p}{)}\PY{p}{,} \PY{n}{method}\PY{o}{=}\PY{l+s}{\PYZdq{}}\PY{l+s}{ML\PYZdq{}}\PY{p}{)}
\PY{n}{s\PYZus{}p2}
\end{Verbatim}
\end{tcolorbox}

    
    \begin{Verbatim}[commandchars=\\\{\}]

Call:
arima(x = x\_p2, order = c(2, 0, 0), method = "ML")

Coefficients:
         ar1      ar2  intercept
      0.8237  -0.1905    -0.0691
s.e.  0.0311   0.0311     0.0861

sigma\^{}2 estimated as 0.9996:  log likelihood = -1419.12,  aic = 2846.23
    \end{Verbatim}

    
    \begin{tcolorbox}[breakable, size=fbox, boxrule=1pt, pad at break*=1mm,colback=cellbackground, colframe=cellborder]
\prompt{In}{incolor}{319}{\boxspacing}
\begin{Verbatim}[commandchars=\\\{\}]
\PY{n}{y}\PY{o}{\PYZlt{}\PYZhy{}}\PY{n+nf}{resid}\PY{p}{(}\PY{n}{s\PYZus{}p1}\PY{p}{)}
\PY{n}{h}\PY{o}{\PYZlt{}\PYZhy{}}\PY{n+nf}{floor}\PY{p}{(}\PY{n+nf}{log}\PY{p}{(}\PY{n}{n}\PY{p}{)}\PY{p}{)}
\PY{n+nf}{Box.test}\PY{p}{(}\PY{n}{x}\PY{p}{,}\PY{n}{lag}\PY{o}{=}\PY{n}{h}\PY{p}{,} \PY{n}{type}\PY{o}{=}\PY{n+nf}{c}\PY{p}{(}\PY{l+s}{\PYZdq{}}\PY{l+s}{Ljung\PYZhy{}Box\PYZdq{}}\PY{p}{)}\PY{p}{)}
\PY{n+nf}{Box.test}\PY{p}{(}\PY{n}{y}\PY{p}{,}\PY{n}{lag}\PY{o}{=}\PY{n}{h}\PY{p}{,}\PY{n}{type}\PY{o}{=}\PY{n+nf}{c}\PY{p}{(}\PY{l+s}{\PYZdq{}}\PY{l+s}{Ljung\PYZhy{}Box\PYZdq{}}\PY{p}{)}\PY{p}{)}

\PY{n+nf}{par}\PY{p}{(}\PY{n}{mfrow} \PY{o}{=} \PY{n+nf}{c}\PY{p}{(}\PY{l+m}{3}\PY{p}{,} \PY{l+m}{1}\PY{p}{)}\PY{p}{)}
\PY{n+nf}{plot}\PY{p}{(}\PY{n}{y}\PY{p}{,}\PY{n}{type}\PY{o}{=}\PY{l+s}{\PYZdq{}}\PY{l+s}{l\PYZdq{}}\PY{p}{,} \PY{n}{main}\PY{o}{=}\PY{l+s}{\PYZdq{}}\PY{l+s}{Residual Data\PYZdq{}}\PY{p}{)}
\PY{n+nf}{hist}\PY{p}{(}\PY{n}{y}\PY{p}{,}\PY{n}{freq}\PY{o}{=}\PY{n+nb+bp}{F}\PY{p}{,} \PY{n}{main}\PY{o}{=}\PY{l+s}{\PYZdq{}}\PY{l+s}{Residual Histogram\PYZdq{}}\PY{p}{)}
\PY{n+nf}{acf}\PY{p}{(}\PY{n}{y}\PY{p}{,} \PY{n}{main}\PY{o}{=}\PY{l+s}{\PYZdq{}}\PY{l+s}{Residual Correlogram\PYZdq{}}\PY{p}{)}
\PY{n+nf}{par}\PY{p}{(}\PY{n}{mfrow} \PY{o}{=} \PY{n+nf}{c}\PY{p}{(}\PY{l+m}{1}\PY{p}{,} \PY{l+m}{1}\PY{p}{)}\PY{p}{)}
\end{Verbatim}
\end{tcolorbox}

    
    \begin{Verbatim}[commandchars=\\\{\}]

	Box-Ljung test

data:  x
X-squared = 4947.7, df = 6, p-value < 2.2e-16

    \end{Verbatim}

    
    
    \begin{Verbatim}[commandchars=\\\{\}]

	Box-Ljung test

data:  y
X-squared = 10.186, df = 6, p-value = 0.117

    \end{Verbatim}

    
    \begin{center}
    \adjustimage{max size={0.9\linewidth}{0.9\paperheight}}{output_50_2.png}
    \end{center}
    { \hspace*{\fill} \\}
    
    \begin{tcolorbox}[breakable, size=fbox, boxrule=1pt, pad at break*=1mm,colback=cellbackground, colframe=cellborder]
\prompt{In}{incolor}{320}{\boxspacing}
\begin{Verbatim}[commandchars=\\\{\}]
\PY{n}{y}\PY{o}{\PYZlt{}\PYZhy{}}\PY{n+nf}{resid}\PY{p}{(}\PY{n}{s\PYZus{}p2}\PY{p}{)}
\PY{n}{h}\PY{o}{\PYZlt{}\PYZhy{}}\PY{n+nf}{floor}\PY{p}{(}\PY{n+nf}{log}\PY{p}{(}\PY{n}{n}\PY{p}{)}\PY{p}{)}
\PY{n+nf}{Box.test}\PY{p}{(}\PY{n}{x}\PY{p}{,}\PY{n}{lag}\PY{o}{=}\PY{n}{h}\PY{p}{,} \PY{n}{type}\PY{o}{=}\PY{n+nf}{c}\PY{p}{(}\PY{l+s}{\PYZdq{}}\PY{l+s}{Ljung\PYZhy{}Box\PYZdq{}}\PY{p}{)}\PY{p}{)}
\PY{n+nf}{Box.test}\PY{p}{(}\PY{n}{y}\PY{p}{,}\PY{n}{lag}\PY{o}{=}\PY{n}{h}\PY{p}{,}\PY{n}{type}\PY{o}{=}\PY{n+nf}{c}\PY{p}{(}\PY{l+s}{\PYZdq{}}\PY{l+s}{Ljung\PYZhy{}Box\PYZdq{}}\PY{p}{)}\PY{p}{)}

\PY{n+nf}{par}\PY{p}{(}\PY{n}{mfrow} \PY{o}{=} \PY{n+nf}{c}\PY{p}{(}\PY{l+m}{3}\PY{p}{,} \PY{l+m}{1}\PY{p}{)}\PY{p}{)}
\PY{n+nf}{plot}\PY{p}{(}\PY{n}{y}\PY{p}{,}\PY{n}{type}\PY{o}{=}\PY{l+s}{\PYZdq{}}\PY{l+s}{l\PYZdq{}}\PY{p}{,} \PY{n}{main}\PY{o}{=}\PY{l+s}{\PYZdq{}}\PY{l+s}{Residual Data\PYZdq{}}\PY{p}{)}
\PY{n+nf}{hist}\PY{p}{(}\PY{n}{y}\PY{p}{,}\PY{n}{freq}\PY{o}{=}\PY{n+nb+bp}{F}\PY{p}{,} \PY{n}{main}\PY{o}{=}\PY{l+s}{\PYZdq{}}\PY{l+s}{Residual Histogram\PYZdq{}}\PY{p}{)}
\PY{n+nf}{acf}\PY{p}{(}\PY{n}{y}\PY{p}{,} \PY{n}{main}\PY{o}{=}\PY{l+s}{\PYZdq{}}\PY{l+s}{Residual Correlogram\PYZdq{}}\PY{p}{)}
\PY{n+nf}{par}\PY{p}{(}\PY{n}{mfrow} \PY{o}{=} \PY{n+nf}{c}\PY{p}{(}\PY{l+m}{1}\PY{p}{,} \PY{l+m}{1}\PY{p}{)}\PY{p}{)}
\end{Verbatim}
\end{tcolorbox}

    
    \begin{Verbatim}[commandchars=\\\{\}]

	Box-Ljung test

data:  x
X-squared = 4947.7, df = 6, p-value < 2.2e-16

    \end{Verbatim}

    
    
    \begin{Verbatim}[commandchars=\\\{\}]

	Box-Ljung test

data:  y
X-squared = 7.1279, df = 6, p-value = 0.3092

    \end{Verbatim}

    
    \begin{center}
    \adjustimage{max size={0.9\linewidth}{0.9\paperheight}}{output_51_2.png}
    \end{center}
    { \hspace*{\fill} \\}
    
    As we can see, residuals for both simulations are IID noise as the
correlogram indicate.

    \#Exercise 4 \textbf{\emph{Simulate an ARMA (2,1). Compute the
autocorrealtion and the partial autocorrelation. Fit the best ARMA
model. Validate it. Make the graphical representation of the
forecasting.}}

    Firstly, let's simulate 1000 points with an ARMA(2,1) model with
parameters \(AR=(0.25, 0.15)\) and \(MA=(0.35)\).

    \begin{tcolorbox}[breakable, size=fbox, boxrule=1pt, pad at break*=1mm,colback=cellbackground, colframe=cellborder]
\prompt{In}{incolor}{321}{\boxspacing}
\begin{Verbatim}[commandchars=\\\{\}]
\PY{n}{ARMA\PYZus{}sim}\PY{o}{\PYZlt{}\PYZhy{}}\PY{n+nf}{arima.sim}\PY{p}{(}\PY{n+nf}{list}\PY{p}{(}\PY{n}{order}\PY{o}{=}\PY{n+nf}{c}\PY{p}{(}\PY{l+m}{2}\PY{p}{,}\PY{l+m}{0}\PY{p}{,}\PY{l+m}{1}\PY{p}{)}\PY{p}{,} \PY{n}{ar}\PY{o}{=}\PY{n+nf}{c}\PY{p}{(}\PY{l+m}{0.25}\PY{p}{,}\PY{l+m}{0.15}\PY{p}{)}\PY{p}{,} \PY{n}{ma}\PY{o}{=}\PY{l+m}{0.35}\PY{p}{)}\PY{p}{,} \PY{n}{n}\PY{o}{=}\PY{l+m}{1000}\PY{p}{)}
\PY{n+nf}{plot}\PY{p}{(}\PY{n}{ARMA\PYZus{}sim}\PY{p}{)}
\end{Verbatim}
\end{tcolorbox}

    \begin{center}
    \adjustimage{max size={0.9\linewidth}{0.9\paperheight}}{output_55_0.png}
    \end{center}
    { \hspace*{\fill} \\}
    
    Now for the ACF and PACF we will use \(k = ln(n) = 6.9\) approximately
7.

    \begin{tcolorbox}[breakable, size=fbox, boxrule=1pt, pad at break*=1mm,colback=cellbackground, colframe=cellborder]
\prompt{In}{incolor}{322}{\boxspacing}
\begin{Verbatim}[commandchars=\\\{\}]
\PY{n+nf}{par}\PY{p}{(}\PY{n}{mfrow} \PY{o}{=} \PY{n+nf}{c}\PY{p}{(}\PY{l+m}{2}\PY{p}{,} \PY{l+m}{1}\PY{p}{)}\PY{p}{)}
\PY{n+nf}{acf}\PY{p}{(}\PY{n}{ARMA\PYZus{}sim}\PY{p}{,}\PY{l+m}{7}\PY{p}{,} \PY{n}{main}\PY{o}{=}\PY{l+s}{\PYZdq{}}\PY{l+s}{Autocorrelation of simulation\PYZdq{}}\PY{p}{)}
\PY{n+nf}{pacf}\PY{p}{(}\PY{n}{ARMA\PYZus{}sim}\PY{p}{,}\PY{l+m}{7}\PY{p}{,} \PY{n}{main}\PY{o}{=}\PY{l+s}{\PYZdq{}}\PY{l+s}{Partial Autocorrelation of simulation\PYZdq{}}\PY{p}{)}
\PY{n+nf}{par}\PY{p}{(}\PY{n}{mfrow} \PY{o}{=} \PY{n+nf}{c}\PY{p}{(}\PY{l+m}{1}\PY{p}{,} \PY{l+m}{1}\PY{p}{)}\PY{p}{)}
\end{Verbatim}
\end{tcolorbox}

    \begin{center}
    \adjustimage{max size={0.9\linewidth}{0.9\paperheight}}{output_57_0.png}
    \end{center}
    { \hspace*{\fill} \\}
    
    Now we can fit the ARMA(2,1) model to the simulation.

    \begin{tcolorbox}[breakable, size=fbox, boxrule=1pt, pad at break*=1mm,colback=cellbackground, colframe=cellborder]
\prompt{In}{incolor}{323}{\boxspacing}
\begin{Verbatim}[commandchars=\\\{\}]
\PY{n}{ARMA\PYZus{}model}\PY{o}{\PYZlt{}\PYZhy{}}\PY{n+nf}{arima}\PY{p}{(}\PY{n}{ARMA\PYZus{}sim}\PY{p}{,} \PY{n}{order}\PY{o}{=}\PY{n+nf}{c}\PY{p}{(}\PY{l+m}{2}\PY{p}{,}\PY{l+m}{0}\PY{p}{,}\PY{l+m}{1}\PY{p}{)}\PY{p}{,} \PY{n}{method}\PY{o}{=}\PY{l+s}{\PYZdq{}}\PY{l+s}{ML\PYZdq{}}\PY{p}{)}
\PY{n}{ARMA\PYZus{}model}
\end{Verbatim}
\end{tcolorbox}

    
    \begin{Verbatim}[commandchars=\\\{\}]

Call:
arima(x = ARMA\_sim, order = c(2, 0, 1), method = "ML")

Coefficients:
          ar1     ar2     ma1  intercept
      -0.2586  0.5184  0.8365     0.0749
s.e.   0.1167  0.0673  0.1185     0.0768

sigma\^{}2 estimated as 0.9622:  log likelihood = -1399.87,  aic = 2809.73
    \end{Verbatim}

    
    \begin{tcolorbox}[breakable, size=fbox, boxrule=1pt, pad at break*=1mm,colback=cellbackground, colframe=cellborder]
\prompt{In}{incolor}{324}{\boxspacing}
\begin{Verbatim}[commandchars=\\\{\}]
\PY{n+nf}{Box.test}\PY{p}{(}\PY{n}{ARMA\PYZus{}sim}\PY{p}{,}\PY{n}{lag}\PY{o}{=}\PY{l+m}{7}\PY{p}{,}\PY{n}{type}\PY{o}{=}\PY{n+nf}{c}\PY{p}{(}\PY{l+s}{\PYZdq{}}\PY{l+s}{Ljung\PYZhy{}Box\PYZdq{}}\PY{p}{)}\PY{p}{)}
\end{Verbatim}
\end{tcolorbox}

    
    \begin{Verbatim}[commandchars=\\\{\}]

	Box-Ljung test

data:  ARMA\_sim
X-squared = 519.65, df = 7, p-value < 2.2e-16

    \end{Verbatim}

    
    \begin{tcolorbox}[breakable, size=fbox, boxrule=1pt, pad at break*=1mm,colback=cellbackground, colframe=cellborder]
\prompt{In}{incolor}{325}{\boxspacing}
\begin{Verbatim}[commandchars=\\\{\}]
\PY{n}{y}\PY{o}{\PYZlt{}\PYZhy{}}\PY{n+nf}{resid}\PY{p}{(}\PY{n}{ARMA\PYZus{}model}\PY{p}{)}
\end{Verbatim}
\end{tcolorbox}

    \begin{tcolorbox}[breakable, size=fbox, boxrule=1pt, pad at break*=1mm,colback=cellbackground, colframe=cellborder]
\prompt{In}{incolor}{326}{\boxspacing}
\begin{Verbatim}[commandchars=\\\{\}]
\PY{n+nf}{Box.test}\PY{p}{(}\PY{n}{y}\PY{p}{,}\PY{n}{lag}\PY{o}{=}\PY{l+m}{7}\PY{p}{,}\PY{n}{type}\PY{o}{=}\PY{n+nf}{c}\PY{p}{(}\PY{l+s}{\PYZdq{}}\PY{l+s}{Ljung\PYZhy{}Box\PYZdq{}}\PY{p}{)}\PY{p}{)}
\end{Verbatim}
\end{tcolorbox}

    
    \begin{Verbatim}[commandchars=\\\{\}]

	Box-Ljung test

data:  y
X-squared = 2.7099, df = 7, p-value = 0.9105

    \end{Verbatim}

    
    \begin{tcolorbox}[breakable, size=fbox, boxrule=1pt, pad at break*=1mm,colback=cellbackground, colframe=cellborder]
\prompt{In}{incolor}{327}{\boxspacing}
\begin{Verbatim}[commandchars=\\\{\}]
\PY{n+nf}{tsdiag}\PY{p}{(}\PY{n}{ARMA\PYZus{}model}\PY{p}{,} \PY{n}{gof.lag}\PY{o}{=}\PY{l+m}{20}\PY{p}{)}
\end{Verbatim}
\end{tcolorbox}

    \begin{center}
    \adjustimage{max size={0.9\linewidth}{0.9\paperheight}}{output_63_0.png}
    \end{center}
    { \hspace*{\fill} \\}
    
    We can see that the residuals are IID noise since the p-values and
correlograms fall into the IID noise patterns.

Let's do forecasting now

    \begin{tcolorbox}[breakable, size=fbox, boxrule=1pt, pad at break*=1mm,colback=cellbackground, colframe=cellborder]
\prompt{In}{incolor}{328}{\boxspacing}
\begin{Verbatim}[commandchars=\\\{\}]
\PY{n+nf}{install.packages}\PY{p}{(}\PY{l+s}{\PYZsq{}}\PY{l+s}{forecast\PYZsq{}}\PY{p}{)}
\end{Verbatim}
\end{tcolorbox}

    \begin{Verbatim}[commandchars=\\\{\}]
Installing package into ‘/usr/local/lib/R/site-library’
(as ‘lib’ is unspecified)

also installing the dependencies ‘lmtest’, ‘Rcpp’, ‘timeDate’, ‘urca’,
‘RcppArmadillo’


    \end{Verbatim}

    \begin{tcolorbox}[breakable, size=fbox, boxrule=1pt, pad at break*=1mm,colback=cellbackground, colframe=cellborder]
\prompt{In}{incolor}{329}{\boxspacing}
\begin{Verbatim}[commandchars=\\\{\}]
\PY{n+nf}{require}\PY{p}{(}\PY{n}{forecast}\PY{p}{)}
\PY{n}{y}\PY{o}{\PYZlt{}\PYZhy{}}\PY{n+nf}{auto.arima}\PY{p}{(}\PY{n}{ARMA\PYZus{}sim}\PY{p}{,} \PY{n}{max.p}\PY{o}{=}\PY{l+m}{2}\PY{p}{,} \PY{n}{max.d}\PY{o}{=}\PY{l+m}{0}\PY{p}{,}\PY{n}{max.q}\PY{o}{=}\PY{l+m}{2}\PY{p}{)}
\PY{n}{y}
\end{Verbatim}
\end{tcolorbox}

    \begin{Verbatim}[commandchars=\\\{\}]
Loading required package: forecast

    \end{Verbatim}

    
    \begin{Verbatim}[commandchars=\\\{\}]
Series: ARMA\_sim 
ARIMA(1,0,0) with zero mean 

Coefficients:
         ar1
      0.5926
s.e.  0.0254

sigma\^{}2 = 0.9661:  log likelihood = -1401.4
AIC=2806.81   AICc=2806.82   BIC=2816.62
    \end{Verbatim}

    
    \begin{tcolorbox}[breakable, size=fbox, boxrule=1pt, pad at break*=1mm,colback=cellbackground, colframe=cellborder]
\prompt{In}{incolor}{330}{\boxspacing}
\begin{Verbatim}[commandchars=\\\{\}]
\PY{n}{yz}\PY{o}{\PYZlt{}\PYZhy{}}\PY{n+nf}{resid}\PY{p}{(}\PY{n}{y}\PY{p}{)}
\PY{n+nf}{Box.test}\PY{p}{(}\PY{n}{yz}\PY{p}{,} \PY{n}{lag}\PY{o}{=}\PY{l+m}{7}\PY{p}{,} \PY{n}{type}\PY{o}{=}\PY{n+nf}{c}\PY{p}{(}\PY{l+s}{\PYZdq{}}\PY{l+s}{Ljung\PYZhy{}Box\PYZdq{}}\PY{p}{)}\PY{p}{)}
\PY{n}{xfc}\PY{o}{\PYZlt{}\PYZhy{}}\PY{n+nf}{forecast}\PY{p}{(}\PY{n}{ARMA\PYZus{}sim}\PY{p}{)}
\PY{n+nf}{plot}\PY{p}{(}\PY{n}{xfc}\PY{p}{,} \PY{n}{type}\PY{o}{=}\PY{l+s}{\PYZdq{}}\PY{l+s}{l\PYZdq{}}\PY{p}{)}
\end{Verbatim}
\end{tcolorbox}

    
    \begin{Verbatim}[commandchars=\\\{\}]

	Box-Ljung test

data:  yz
X-squared = 5.0994, df = 7, p-value = 0.6478

    \end{Verbatim}

    
    \begin{center}
    \adjustimage{max size={0.9\linewidth}{0.9\paperheight}}{output_67_1.png}
    \end{center}
    { \hspace*{\fill} \\}
    
    \#Exercise 5 \textbf{\emph{Take the file Nile in datasets. Fit the best
ARIMA model to the process. Validate it. Make the graphical
representation of the forecasting.}}

    Let's load the Nile dataset into our workspace.

    \begin{tcolorbox}[breakable, size=fbox, boxrule=1pt, pad at break*=1mm,colback=cellbackground, colframe=cellborder]
\prompt{In}{incolor}{439}{\boxspacing}
\begin{Verbatim}[commandchars=\\\{\}]
\PY{n}{nile\PYZus{}data} \PY{o}{\PYZlt{}\PYZhy{}} \PY{n}{Nile}
\PY{n}{nile\PYZus{}data}
\end{Verbatim}
\end{tcolorbox}

    A Time Series:\\\begin{enumerate*}
\item 1120
\item 1160
\item 963
\item 1210
\item 1160
\item 1160
\item 813
\item 1230
\item 1370
\item 1140
\item 995
\item 935
\item 1110
\item 994
\item 1020
\item 960
\item 1180
\item 799
\item 958
\item 1140
\item 1100
\item 1210
\item 1150
\item 1250
\item 1260
\item 1220
\item 1030
\item 1100
\item 774
\item 840
\item 874
\item 694
\item 940
\item 833
\item 701
\item 916
\item 692
\item 1020
\item 1050
\item 969
\item 831
\item 726
\item 456
\item 824
\item 702
\item 1120
\item 1100
\item 832
\item 764
\item 821
\item 768
\item 845
\item 864
\item 862
\item 698
\item 845
\item 744
\item 796
\item 1040
\item 759
\item 781
\item 865
\item 845
\item 944
\item 984
\item 897
\item 822
\item 1010
\item 771
\item 676
\item 649
\item 846
\item 812
\item 742
\item 801
\item 1040
\item 860
\item 874
\item 848
\item 890
\item 744
\item 749
\item 838
\item 1050
\item 918
\item 986
\item 797
\item 923
\item 975
\item 815
\item 1020
\item 906
\item 901
\item 1170
\item 912
\item 746
\item 919
\item 718
\item 714
\item 740
\end{enumerate*}


    
    And now let's find the best ARIMA model to the data.

    \begin{tcolorbox}[breakable, size=fbox, boxrule=1pt, pad at break*=1mm,colback=cellbackground, colframe=cellborder]
\prompt{In}{incolor}{440}{\boxspacing}
\begin{Verbatim}[commandchars=\\\{\}]
\PY{n}{ARMA\PYZus{}nile}\PY{o}{\PYZlt{}\PYZhy{}}\PY{n+nf}{auto.arima}\PY{p}{(}\PY{n}{nile\PYZus{}data}\PY{p}{)}
\PY{n}{ARMA\PYZus{}nile}
\end{Verbatim}
\end{tcolorbox}

    
    \begin{Verbatim}[commandchars=\\\{\}]
Series: nile\_data 
ARIMA(1,1,1) 

Coefficients:
         ar1      ma1
      0.2544  -0.8741
s.e.  0.1194   0.0605

sigma\^{}2 = 20177:  log likelihood = -630.63
AIC=1267.25   AICc=1267.51   BIC=1275.04
    \end{Verbatim}

    
    \begin{tcolorbox}[breakable, size=fbox, boxrule=1pt, pad at break*=1mm,colback=cellbackground, colframe=cellborder]
\prompt{In}{incolor}{443}{\boxspacing}
\begin{Verbatim}[commandchars=\\\{\}]
\PY{n+nf}{par}\PY{p}{(}\PY{n}{mfrow} \PY{o}{=} \PY{n+nf}{c}\PY{p}{(}\PY{l+m}{2}\PY{p}{,} \PY{l+m}{1}\PY{p}{)}\PY{p}{)}
\PY{n}{yz}\PY{o}{\PYZlt{}\PYZhy{}}\PY{n+nf}{resid}\PY{p}{(}\PY{n}{ARMA\PYZus{}nile}\PY{p}{)}
\PY{n+nf}{plot}\PY{p}{(}\PY{n}{yz}\PY{p}{,} \PY{n}{main}\PY{o}{=}\PY{l+s}{\PYZdq{}}\PY{l+s}{Residuals of the model\PYZdq{}}\PY{p}{)}
\PY{n+nf}{acf}\PY{p}{(}\PY{n}{yz}\PY{p}{,} \PY{n}{main}\PY{o}{=}\PY{l+s}{\PYZdq{}}\PY{l+s}{ACF of the Residuals\PYZdq{}}\PY{p}{)}
\PY{n+nf}{Box.test}\PY{p}{(}\PY{n}{yz}\PY{p}{,} \PY{n}{type}\PY{o}{=}\PY{n+nf}{c}\PY{p}{(}\PY{l+s}{\PYZdq{}}\PY{l+s}{Ljung\PYZhy{}Box\PYZdq{}}\PY{p}{)}\PY{p}{)}
\PY{n+nf}{par}\PY{p}{(}\PY{n}{mfrow} \PY{o}{=} \PY{n+nf}{c}\PY{p}{(}\PY{l+m}{1}\PY{p}{,} \PY{l+m}{1}\PY{p}{)}\PY{p}{)}
\end{Verbatim}
\end{tcolorbox}

    
    \begin{Verbatim}[commandchars=\\\{\}]

	Box-Ljung test

data:  yz
X-squared = 0.10739, df = 1, p-value = 0.7431

    \end{Verbatim}

    
    \begin{center}
    \adjustimage{max size={0.9\linewidth}{0.9\paperheight}}{output_73_1.png}
    \end{center}
    { \hspace*{\fill} \\}
    
    As we can see, the residuals are IID noise, so the model is a good fit
to the data.

    \begin{tcolorbox}[breakable, size=fbox, boxrule=1pt, pad at break*=1mm,colback=cellbackground, colframe=cellborder]
\prompt{In}{incolor}{334}{\boxspacing}
\begin{Verbatim}[commandchars=\\\{\}]
\PY{n}{xfc}\PY{o}{\PYZlt{}\PYZhy{}}\PY{n+nf}{forecast}\PY{p}{(}\PY{n}{nile\PYZus{}data}\PY{p}{)}
\PY{n+nf}{plot}\PY{p}{(}\PY{n}{xfc}\PY{p}{,} \PY{n}{type}\PY{o}{=}\PY{l+s}{\PYZdq{}}\PY{l+s}{l\PYZdq{}}\PY{p}{)}
\end{Verbatim}
\end{tcolorbox}

    \begin{center}
    \adjustimage{max size={0.9\linewidth}{0.9\paperheight}}{output_75_0.png}
    \end{center}
    { \hspace*{\fill} \\}
    
    \#Exercise 6 \textbf{\emph{Simulate a FARIMA time series. Fit it the
best model and test that the residuals of the fitted model are a white
noise. Fit a FARIMA model to Nile data in datasets. Check that the
fitted model is a good model.}}

    \begin{tcolorbox}[breakable, size=fbox, boxrule=1pt, pad at break*=1mm,colback=cellbackground, colframe=cellborder]
\prompt{In}{incolor}{335}{\boxspacing}
\begin{Verbatim}[commandchars=\\\{\}]
\PY{n+nf}{install.packages}\PY{p}{(}\PY{l+s}{\PYZdq{}}\PY{l+s}{fracdiff\PYZdq{}}\PY{p}{)}
\end{Verbatim}
\end{tcolorbox}

    \begin{Verbatim}[commandchars=\\\{\}]
Installing package into ‘/usr/local/lib/R/site-library’
(as ‘lib’ is unspecified)

    \end{Verbatim}

    To begin, let's simulate 1000 data points with an AR of 0.9 and a \(d\)
parameter equal to 0.4.

    \begin{tcolorbox}[breakable, size=fbox, boxrule=1pt, pad at break*=1mm,colback=cellbackground, colframe=cellborder]
\prompt{In}{incolor}{445}{\boxspacing}
\begin{Verbatim}[commandchars=\\\{\}]
\PY{n}{n}\PY{o}{\PYZlt{}\PYZhy{}}\PY{l+m}{1000}
\PY{n}{sample}\PY{o}{\PYZlt{}\PYZhy{}}\PY{n+nf}{fracdiff.sim}\PY{p}{(}\PY{n}{n}\PY{p}{,} \PY{n}{ar}\PY{o}{=}\PY{l+m}{0.9}\PY{p}{,} \PY{n}{d}\PY{o}{=}\PY{l+m}{0.4}\PY{p}{)}
\PY{n}{x}\PY{o}{\PYZlt{}\PYZhy{}}\PY{n}{sample}\PY{o}{\PYZdl{}}\PY{n}{series}
\PY{n+nf}{plot}\PY{p}{(}\PY{n}{x}\PY{p}{,} \PY{n}{type}\PY{o}{=}\PY{l+s}{\PYZdq{}}\PY{l+s}{l\PYZdq{}}\PY{p}{,} \PY{n}{main}\PY{o}{=}\PY{l+s}{\PYZdq{}}\PY{l+s}{Simulation with AR=0.9 and d=0.4\PYZdq{}}\PY{p}{)}
\end{Verbatim}
\end{tcolorbox}

    \begin{center}
    \adjustimage{max size={0.9\linewidth}{0.9\paperheight}}{output_79_0.png}
    \end{center}
    { \hspace*{\fill} \\}
    
    Now we can fit the ARFIMA model to the simulation and get the
coefficients.

    \begin{tcolorbox}[breakable, size=fbox, boxrule=1pt, pad at break*=1mm,colback=cellbackground, colframe=cellborder]
\prompt{In}{incolor}{337}{\boxspacing}
\begin{Verbatim}[commandchars=\\\{\}]
\PY{n+nf}{library}\PY{p}{(}\PY{n}{fracdiff}\PY{p}{)}
\PY{n}{fit1}\PY{o}{\PYZlt{}\PYZhy{}}\PY{n+nf}{fracdiff}\PY{p}{(}\PY{n}{x}\PY{p}{,} \PY{n}{nar}\PY{o}{=}\PY{l+m}{1}\PY{p}{,} \PY{n}{drange}\PY{o}{=}\PY{n+nf}{c}\PY{p}{(}\PY{l+m}{0}\PY{p}{,}\PY{l+m}{0.5}\PY{p}{)}\PY{p}{)}
\PY{n+nf}{summary}\PY{p}{(}\PY{n}{fit1}\PY{p}{)}
\end{Verbatim}
\end{tcolorbox}

    
    \begin{Verbatim}[commandchars=\\\{\}]

Call:
  fracdiff(x = x, nar = 1, drange = c(0, 0.5)) 

Coefficients:
   Estimate Std. Error z value Pr(>|z|)    
d   0.35647    0.01191   29.94   <2e-16 ***
ar  0.92014    0.01317   69.87   <2e-16 ***
---
Signif. codes:  0 ‘***’ 0.001 ‘**’ 0.01 ‘*’ 0.05 ‘.’ 0.1 ‘ ’ 1
sigma[eps] = 1.001226 
[d.tol = 0.0001221, M = 100, h = 1.498e-05]
Log likelihood: -1421 ==> AIC = 2847.348 [3 deg.freedom]
    \end{Verbatim}

    
    \begin{tcolorbox}[breakable, size=fbox, boxrule=1pt, pad at break*=1mm,colback=cellbackground, colframe=cellborder]
\prompt{In}{incolor}{338}{\boxspacing}
\begin{Verbatim}[commandchars=\\\{\}]
\PY{n}{d}\PY{o}{\PYZlt{}\PYZhy{}}\PY{n}{fit1}\PY{o}{\PYZdl{}}\PY{n}{d}
\PY{n}{d}
\end{Verbatim}
\end{tcolorbox}

    0.356470668074661

    
    We see that both parameters are resonably close with the \(d\) equal to
0.36, close to 0.4. And the parameter of the AR model is approximately
0.9 as well.

    \begin{tcolorbox}[breakable, size=fbox, boxrule=1pt, pad at break*=1mm,colback=cellbackground, colframe=cellborder]
\prompt{In}{incolor}{446}{\boxspacing}
\begin{Verbatim}[commandchars=\\\{\}]
\PY{n+nf}{par}\PY{p}{(}\PY{n}{mfrow} \PY{o}{=} \PY{n+nf}{c}\PY{p}{(}\PY{l+m}{2}\PY{p}{,} \PY{l+m}{1}\PY{p}{)}\PY{p}{)}
\PY{n}{fit2}\PY{o}{\PYZlt{}\PYZhy{}}\PY{n+nf}{diffseries}\PY{p}{(}\PY{n}{x}\PY{p}{,}\PY{n}{d}\PY{p}{)}
\PY{n+nf}{plot}\PY{p}{(}\PY{n}{fit2}\PY{p}{,} \PY{n}{type}\PY{o}{=}\PY{l+s}{\PYZdq{}}\PY{l+s}{l\PYZdq{}}\PY{p}{,} \PY{n}{main}\PY{o}{=}\PY{l+s}{\PYZdq{}}\PY{l+s}{Diffseries of the data\PYZdq{}}\PY{p}{)}
\PY{n+nf}{acf}\PY{p}{(}\PY{n}{fit2}\PY{p}{,} \PY{n}{main}\PY{o}{=}\PY{l+s}{\PYZdq{}}\PY{l+s}{ACF of the diffseries\PYZdq{}}\PY{p}{)}
\PY{n+nf}{par}\PY{p}{(}\PY{n}{mfrow} \PY{o}{=} \PY{n+nf}{c}\PY{p}{(}\PY{l+m}{1}\PY{p}{,} \PY{l+m}{1}\PY{p}{)}\PY{p}{)}
\end{Verbatim}
\end{tcolorbox}

    \begin{center}
    \adjustimage{max size={0.9\linewidth}{0.9\paperheight}}{output_84_0.png}
    \end{center}
    { \hspace*{\fill} \\}
    
    \begin{tcolorbox}[breakable, size=fbox, boxrule=1pt, pad at break*=1mm,colback=cellbackground, colframe=cellborder]
\prompt{In}{incolor}{340}{\boxspacing}
\begin{Verbatim}[commandchars=\\\{\}]
\PY{n}{fit3}\PY{o}{\PYZlt{}\PYZhy{}}\PY{n+nf}{arima}\PY{p}{(}\PY{n}{fit2}\PY{p}{,}\PY{n}{order}\PY{o}{=}\PY{n+nf}{c}\PY{p}{(}\PY{l+m}{1}\PY{p}{,}\PY{l+m}{0}\PY{p}{,}\PY{l+m}{0}\PY{p}{)}\PY{p}{)}
\PY{n}{z}\PY{o}{\PYZlt{}\PYZhy{}}\PY{n+nf}{resid}\PY{p}{(}\PY{n}{fit3}\PY{p}{)}
\PY{n+nf}{acf}\PY{p}{(}\PY{n}{z}\PY{p}{)}
\PY{n+nf}{Box.test}\PY{p}{(}\PY{n}{z}\PY{p}{,}\PY{n+nf}{log}\PY{p}{(}\PY{n}{n}\PY{p}{)}\PY{p}{,}\PY{n}{type}\PY{o}{=}\PY{n+nf}{c}\PY{p}{(}\PY{l+s}{\PYZdq{}}\PY{l+s}{Ljung\PYZhy{}Box\PYZdq{}}\PY{p}{)}\PY{p}{)}
\end{Verbatim}
\end{tcolorbox}

    
    \begin{Verbatim}[commandchars=\\\{\}]

	Box-Ljung test

data:  z
X-squared = 4.6285, df = 6.9078, p-value = 0.6956

    \end{Verbatim}

    
    \begin{center}
    \adjustimage{max size={0.9\linewidth}{0.9\paperheight}}{output_85_1.png}
    \end{center}
    { \hspace*{\fill} \\}
    
    We can see that this model is a good fit since the resulting residuals
are IID noise.

Now let us test it on the Nile Dataset.

    \begin{tcolorbox}[breakable, size=fbox, boxrule=1pt, pad at break*=1mm,colback=cellbackground, colframe=cellborder]
\prompt{In}{incolor}{447}{\boxspacing}
\begin{Verbatim}[commandchars=\\\{\}]
\PY{n}{nile\PYZus{}data} \PY{o}{\PYZlt{}\PYZhy{}} \PY{n}{Nile} \PY{c+c1}{\PYZsh{} Load the data}
\PY{n}{nile\PYZus{}data}
\end{Verbatim}
\end{tcolorbox}

    A Time Series:\\\begin{enumerate*}
\item 1120
\item 1160
\item 963
\item 1210
\item 1160
\item 1160
\item 813
\item 1230
\item 1370
\item 1140
\item 995
\item 935
\item 1110
\item 994
\item 1020
\item 960
\item 1180
\item 799
\item 958
\item 1140
\item 1100
\item 1210
\item 1150
\item 1250
\item 1260
\item 1220
\item 1030
\item 1100
\item 774
\item 840
\item 874
\item 694
\item 940
\item 833
\item 701
\item 916
\item 692
\item 1020
\item 1050
\item 969
\item 831
\item 726
\item 456
\item 824
\item 702
\item 1120
\item 1100
\item 832
\item 764
\item 821
\item 768
\item 845
\item 864
\item 862
\item 698
\item 845
\item 744
\item 796
\item 1040
\item 759
\item 781
\item 865
\item 845
\item 944
\item 984
\item 897
\item 822
\item 1010
\item 771
\item 676
\item 649
\item 846
\item 812
\item 742
\item 801
\item 1040
\item 860
\item 874
\item 848
\item 890
\item 744
\item 749
\item 838
\item 1050
\item 918
\item 986
\item 797
\item 923
\item 975
\item 815
\item 1020
\item 906
\item 901
\item 1170
\item 912
\item 746
\item 919
\item 718
\item 714
\item 740
\end{enumerate*}


    
    \begin{tcolorbox}[breakable, size=fbox, boxrule=1pt, pad at break*=1mm,colback=cellbackground, colframe=cellborder]
\prompt{In}{incolor}{448}{\boxspacing}
\begin{Verbatim}[commandchars=\\\{\}]
\PY{n}{fit\PYZus{}nile}\PY{o}{\PYZlt{}\PYZhy{}}\PY{n+nf}{fracdiff}\PY{p}{(}\PY{n}{nile\PYZus{}data}\PY{p}{,} \PY{n}{nar}\PY{o}{=}\PY{l+m}{1}\PY{p}{,}\PY{n}{nma}\PY{o}{=}\PY{l+m}{1}\PY{p}{)} \PY{c+c1}{\PYZsh{} Fit the ARFIMA model}
\PY{n+nf}{summary}\PY{p}{(}\PY{n}{fit\PYZus{}nile}\PY{p}{)}
\end{Verbatim}
\end{tcolorbox}

    
    \begin{Verbatim}[commandchars=\\\{\}]

Call:
  fracdiff(x = nile\_data, nar = 1, nma = 1) 

Coefficients:
   Estimate Std. Error z value Pr(>|z|)    
d   0.19005    0.01684  11.288   <2e-16 ***
ar  0.90474    0.09397   9.628   <2e-16 ***
ma  0.83657    0.06026  13.882   <2e-16 ***
---
Signif. codes:  0 ‘***’ 0.001 ‘**’ 0.01 ‘*’ 0.05 ‘.’ 0.1 ‘ ’ 1
sigma[eps] = 139.8735 
[d.tol = 0.0001221, M = 100, h = 6.713e-06]
Log likelihood: -636.1 ==> AIC = 1280.181 [4 deg.freedom]
    \end{Verbatim}

    
    \begin{tcolorbox}[breakable, size=fbox, boxrule=1pt, pad at break*=1mm,colback=cellbackground, colframe=cellborder]
\prompt{In}{incolor}{449}{\boxspacing}
\begin{Verbatim}[commandchars=\\\{\}]
\PY{n}{d\PYZus{}nile}\PY{o}{\PYZlt{}\PYZhy{}}\PY{n}{fit\PYZus{}nile}\PY{o}{\PYZdl{}}\PY{n}{d}
\PY{n}{d\PYZus{}nile}
\end{Verbatim}
\end{tcolorbox}

    0.190054053215489

    
    \begin{tcolorbox}[breakable, size=fbox, boxrule=1pt, pad at break*=1mm,colback=cellbackground, colframe=cellborder]
\prompt{In}{incolor}{450}{\boxspacing}
\begin{Verbatim}[commandchars=\\\{\}]
\PY{n+nf}{par}\PY{p}{(}\PY{n}{mfrow} \PY{o}{=} \PY{n+nf}{c}\PY{p}{(}\PY{l+m}{2}\PY{p}{,} \PY{l+m}{1}\PY{p}{)}\PY{p}{)}
\PY{n}{y\PYZus{}nile}\PY{o}{\PYZlt{}\PYZhy{}}\PY{n+nf}{diffseries}\PY{p}{(}\PY{n}{nile\PYZus{}data}\PY{p}{,}\PY{n}{d\PYZus{}nile}\PY{p}{)}
\PY{n+nf}{plot}\PY{p}{(}\PY{n}{y\PYZus{}nile}\PY{p}{,} \PY{n}{type}\PY{o}{=}\PY{l+s}{\PYZdq{}}\PY{l+s}{l\PYZdq{}}\PY{p}{,} \PY{n}{main}\PY{o}{=}\PY{l+s}{\PYZdq{}}\PY{l+s}{Diffseries of Nile\PYZdq{}}\PY{p}{)}
\PY{n+nf}{acf}\PY{p}{(}\PY{n}{y\PYZus{}nile}\PY{p}{)}
\PY{n+nf}{par}\PY{p}{(}\PY{n}{mfrow} \PY{o}{=} \PY{n+nf}{c}\PY{p}{(}\PY{l+m}{1}\PY{p}{,} \PY{l+m}{1}\PY{p}{)}\PY{p}{)}
\end{Verbatim}
\end{tcolorbox}

    \begin{center}
    \adjustimage{max size={0.9\linewidth}{0.9\paperheight}}{output_90_0.png}
    \end{center}
    { \hspace*{\fill} \\}
    
    \begin{tcolorbox}[breakable, size=fbox, boxrule=1pt, pad at break*=1mm,colback=cellbackground, colframe=cellborder]
\prompt{In}{incolor}{345}{\boxspacing}
\begin{Verbatim}[commandchars=\\\{\}]
\PY{n}{yz\PYZus{}nile}\PY{o}{\PYZlt{}\PYZhy{}}\PY{n+nf}{arima}\PY{p}{(}\PY{n}{y\PYZus{}nile}\PY{p}{,}\PY{n}{order}\PY{o}{=}\PY{n+nf}{c}\PY{p}{(}\PY{l+m}{1}\PY{p}{,}\PY{l+m}{0}\PY{p}{,}\PY{l+m}{0}\PY{p}{)}\PY{p}{)}
\PY{n}{z}\PY{o}{\PYZlt{}\PYZhy{}}\PY{n+nf}{resid}\PY{p}{(}\PY{n}{yz\PYZus{}nile}\PY{p}{)}
\PY{n+nf}{acf}\PY{p}{(}\PY{n}{z}\PY{p}{)}
\PY{n+nf}{Box.test}\PY{p}{(}\PY{n}{z}\PY{p}{,}\PY{n+nf}{log}\PY{p}{(}\PY{n+nf}{length}\PY{p}{(}\PY{n}{nile\PYZus{}data}\PY{p}{)}\PY{p}{)}\PY{p}{,}\PY{n}{type}\PY{o}{=}\PY{n+nf}{c}\PY{p}{(}\PY{l+s}{\PYZdq{}}\PY{l+s}{Ljung\PYZhy{}Box\PYZdq{}}\PY{p}{)}\PY{p}{)}
\end{Verbatim}
\end{tcolorbox}

    
    \begin{Verbatim}[commandchars=\\\{\}]

	Box-Ljung test

data:  z
X-squared = 1.9453, df = 4.6052, p-value = 0.8187

    \end{Verbatim}

    
    \begin{center}
    \adjustimage{max size={0.9\linewidth}{0.9\paperheight}}{output_91_1.png}
    \end{center}
    { \hspace*{\fill} \\}
    
    The resulting residuals for this series is IID noise as the model is a
good fit for it.

    \#Exercise 7 \textbf{\emph{Simulate a GARCH(1,1) time series. Fit the
best model to this series. Check that the fitted model it is a good
model. Fit a GARCH model to the logarithmic transformation of series in
EuStockMarkets of datasets. Check the stylized facts (un-correlation,
correlation of the squares, heavy tails, volatility clustering). Check
that the fitted model is a good model.}}

    \begin{tcolorbox}[breakable, size=fbox, boxrule=1pt, pad at break*=1mm,colback=cellbackground, colframe=cellborder]
\prompt{In}{incolor}{346}{\boxspacing}
\begin{Verbatim}[commandchars=\\\{\}]
\PY{n+nf}{par}\PY{p}{(}\PY{n}{mfrow} \PY{o}{=} \PY{n+nf}{c}\PY{p}{(}\PY{l+m}{3}\PY{p}{,} \PY{l+m}{1}\PY{p}{)}\PY{p}{)}
\PY{n}{n}\PY{o}{\PYZlt{}\PYZhy{}}\PY{l+m}{10000}
\PY{n}{a0}\PY{o}{\PYZlt{}\PYZhy{}}\PY{l+m}{0.1}
\PY{n}{a1}\PY{o}{\PYZlt{}\PYZhy{}}\PY{l+m}{0.4}
\PY{n}{b1}\PY{o}{\PYZlt{}\PYZhy{}}\PY{l+m}{0.5}
\PY{n}{w}\PY{o}{\PYZlt{}\PYZhy{}}\PY{n+nf}{rnorm}\PY{p}{(}\PY{n}{n}\PY{p}{)}
\PY{n}{y}\PY{o}{\PYZlt{}\PYZhy{}}\PY{n+nf}{rep}\PY{p}{(}\PY{l+m}{0}\PY{p}{,}\PY{n}{n}\PY{p}{)}
\PY{n}{h}\PY{o}{\PYZlt{}\PYZhy{}}\PY{n+nf}{rep}\PY{p}{(}\PY{l+m}{0}\PY{p}{,}\PY{n}{n}\PY{p}{)}
\PY{n+nf}{for}\PY{p}{(}\PY{n}{i} \PY{n}{in} \PY{l+m}{2}\PY{o}{:}\PY{n}{n}\PY{p}{)}\PY{p}{\PYZob{}}
  \PY{n}{h}\PY{p}{[}\PY{n}{i}\PY{p}{]}\PY{o}{\PYZlt{}\PYZhy{}}\PY{n}{a0}\PY{o}{+}\PY{n}{a1}\PY{o}{*}\PY{p}{(}\PY{n}{y}\PY{p}{[}\PY{n}{i}\PY{l+m}{\PYZhy{}1}\PY{p}{]}\PY{o}{\PYZca{}}\PY{l+m}{2}\PY{p}{)}\PY{o}{+}\PY{n}{b1}\PY{o}{*}\PY{n}{h}\PY{p}{[}\PY{n}{i}\PY{l+m}{\PYZhy{}1}\PY{p}{]}
  \PY{n}{y}\PY{p}{[}\PY{n}{i}\PY{p}{]}\PY{o}{\PYZlt{}\PYZhy{}}\PY{n}{w}\PY{p}{[}\PY{n}{i}\PY{p}{]}\PY{o}{*}\PY{n+nf}{sqrt}\PY{p}{(}\PY{n}{h}\PY{p}{[}\PY{n}{i}\PY{p}{]}\PY{p}{)}
\PY{p}{\PYZcb{}}
\PY{n+nf}{plot}\PY{p}{(}\PY{n}{y}\PY{p}{,} \PY{n}{type}\PY{o}{=}\PY{l+s}{\PYZdq{}}\PY{l+s}{l\PYZdq{}}\PY{p}{)}
\PY{n+nf}{acf}\PY{p}{(}\PY{n}{y}\PY{p}{)}
\PY{n+nf}{acf}\PY{p}{(}\PY{n}{y}\PY{o}{\PYZca{}}\PY{l+m}{2}\PY{p}{)}
\PY{n+nf}{par}\PY{p}{(}\PY{n}{mfrow} \PY{o}{=} \PY{n+nf}{c}\PY{p}{(}\PY{l+m}{1}\PY{p}{,} \PY{l+m}{1}\PY{p}{)}\PY{p}{)}
\end{Verbatim}
\end{tcolorbox}

    \begin{center}
    \adjustimage{max size={0.9\linewidth}{0.9\paperheight}}{output_94_0.png}
    \end{center}
    { \hspace*{\fill} \\}
    
    \begin{tcolorbox}[breakable, size=fbox, boxrule=1pt, pad at break*=1mm,colback=cellbackground, colframe=cellborder]
\prompt{In}{incolor}{347}{\boxspacing}
\begin{Verbatim}[commandchars=\\\{\}]
\PY{n+nf}{install.packages}\PY{p}{(}\PY{l+s}{\PYZdq{}}\PY{l+s}{tseries\PYZdq{}}\PY{p}{)}
\PY{n+nf}{library}\PY{p}{(}\PY{n}{tseries}\PY{p}{)}
\end{Verbatim}
\end{tcolorbox}

    \begin{Verbatim}[commandchars=\\\{\}]
Installing package into ‘/usr/local/lib/R/site-library’
(as ‘lib’ is unspecified)

    \end{Verbatim}

    \begin{tcolorbox}[breakable, size=fbox, boxrule=1pt, pad at break*=1mm,colback=cellbackground, colframe=cellborder]
\prompt{In}{incolor}{348}{\boxspacing}
\begin{Verbatim}[commandchars=\\\{\}]
\PY{n+nf}{par}\PY{p}{(}\PY{n}{mfrow} \PY{o}{=} \PY{n+nf}{c}\PY{p}{(}\PY{l+m}{3}\PY{p}{,} \PY{l+m}{1}\PY{p}{)}\PY{p}{)}
\PY{n}{sp}\PY{o}{\PYZlt{}\PYZhy{}}\PY{n+nf}{garch}\PY{p}{(}\PY{n}{y}\PY{p}{,} \PY{n}{trace}\PY{o}{=}\PY{n+nb+bp}{F}\PY{p}{)}
\PY{n}{res}\PY{o}{\PYZlt{}\PYZhy{}}\PY{n}{sp}\PY{o}{\PYZdl{}}\PY{n}{res}\PY{p}{[}\PY{l+m}{\PYZhy{}1}\PY{p}{]}
\PY{n+nf}{plot}\PY{p}{(}\PY{n}{res}\PY{p}{,}\PY{n}{type}\PY{o}{=}\PY{l+s}{\PYZdq{}}\PY{l+s}{l\PYZdq{}}\PY{p}{)}
\PY{n+nf}{acf}\PY{p}{(}\PY{n}{res}\PY{p}{)}
\PY{n+nf}{acf}\PY{p}{(}\PY{n}{res}\PY{o}{\PYZca{}}\PY{l+m}{2}\PY{p}{)}
\PY{n+nf}{confint}\PY{p}{(}\PY{n}{sp}\PY{p}{)}
\PY{n+nf}{par}\PY{p}{(}\PY{n}{mfrow} \PY{o}{=} \PY{n+nf}{c}\PY{p}{(}\PY{l+m}{1}\PY{p}{,} \PY{l+m}{1}\PY{p}{)}\PY{p}{)}
\end{Verbatim}
\end{tcolorbox}

    A matrix: 3 × 2 of type dbl
\begin{tabular}{r|ll}
  & 2.5 \% & 97.5 \%\\
\hline
	a0 & 0.09267028 & 0.1192382\\
	a1 & 0.37696719 & 0.4470673\\
	b1 & 0.44212258 & 0.5114359\\
\end{tabular}


    
    \begin{center}
    \adjustimage{max size={0.9\linewidth}{0.9\paperheight}}{output_96_1.png}
    \end{center}
    { \hspace*{\fill} \\}
    
    Let us try the EU stock markets dataset:

    \begin{tcolorbox}[breakable, size=fbox, boxrule=1pt, pad at break*=1mm,colback=cellbackground, colframe=cellborder]
\prompt{In}{incolor}{411}{\boxspacing}
\begin{Verbatim}[commandchars=\\\{\}]
\PY{n}{eu\PYZus{}data} \PY{o}{\PYZlt{}\PYZhy{}} \PY{n}{EuStockMarkets}
\PY{n+nf}{length}\PY{p}{(}\PY{n}{eu\PYZus{}data}\PY{p}{)}
\end{Verbatim}
\end{tcolorbox}

    7440

    
    \begin{tcolorbox}[breakable, size=fbox, boxrule=1pt, pad at break*=1mm,colback=cellbackground, colframe=cellborder]
\prompt{In}{incolor}{412}{\boxspacing}
\begin{Verbatim}[commandchars=\\\{\}]
\PY{n}{log\PYZus{}eu} \PY{o}{\PYZlt{}\PYZhy{}} \PY{n+nf}{log}\PY{p}{(}\PY{n}{eu\PYZus{}data}\PY{p}{)}
\PY{n+nf}{summary}\PY{p}{(}\PY{n}{log\PYZus{}eu}\PY{p}{)}
\end{Verbatim}
\end{tcolorbox}

    
    \begin{Verbatim}[commandchars=\\\{\}]
      DAX             SMI             CAC             FTSE      
 Min.   :7.246   Min.   :7.370   Min.   :7.385   Min.   :7.732  
 1st Qu.:7.464   1st Qu.:7.680   1st Qu.:7.536   1st Qu.:7.953  
 Median :7.669   Median :7.936   Median :7.597   Median :8.085  
 Mean   :7.763   Mean   :8.023   Mean   :7.682   Mean   :8.145  
 3rd Qu.:7.909   3rd Qu.:8.246   3rd Qu.:7.729   3rd Qu.:8.292  
 Max.   :8.730   Max.   :9.037   Max.   :8.387   Max.   :8.729  
    \end{Verbatim}

    
    \begin{tcolorbox}[breakable, size=fbox, boxrule=1pt, pad at break*=1mm,colback=cellbackground, colframe=cellborder]
\prompt{In}{incolor}{422}{\boxspacing}
\begin{Verbatim}[commandchars=\\\{\}]
\PY{n}{DAX} \PY{o}{\PYZlt{}\PYZhy{}} \PY{n+nf}{diff}\PY{p}{(}\PY{n}{log\PYZus{}eu}\PY{p}{[}\PY{p}{,} \PY{l+m}{1}\PY{p}{]}\PY{p}{)}
\PY{n}{SMI} \PY{o}{\PYZlt{}\PYZhy{}} \PY{n+nf}{diff}\PY{p}{(}\PY{n}{log\PYZus{}eu}\PY{p}{[}\PY{p}{,} \PY{l+m}{2}\PY{p}{]}\PY{p}{)}
\PY{n}{CAC} \PY{o}{\PYZlt{}\PYZhy{}} \PY{n+nf}{diff}\PY{p}{(}\PY{n}{log\PYZus{}eu}\PY{p}{[}\PY{p}{,} \PY{l+m}{3}\PY{p}{]}\PY{p}{)}
\PY{n}{FTSE} \PY{o}{\PYZlt{}\PYZhy{}} \PY{n+nf}{diff}\PY{p}{(}\PY{n}{log\PYZus{}eu}\PY{p}{[}\PY{p}{,} \PY{l+m}{4}\PY{p}{]}\PY{p}{)}
\end{Verbatim}
\end{tcolorbox}

    \begin{tcolorbox}[breakable, size=fbox, boxrule=1pt, pad at break*=1mm,colback=cellbackground, colframe=cellborder]
\prompt{In}{incolor}{423}{\boxspacing}
\begin{Verbatim}[commandchars=\\\{\}]
\PY{n+nf}{par}\PY{p}{(}\PY{n}{mfrow} \PY{o}{=} \PY{n+nf}{c}\PY{p}{(}\PY{l+m}{3}\PY{p}{,} \PY{l+m}{1}\PY{p}{)}\PY{p}{)}
\PY{n}{sp}\PY{o}{\PYZlt{}\PYZhy{}}\PY{n+nf}{garch}\PY{p}{(}\PY{n}{DAX}\PY{p}{,} \PY{n}{trace}\PY{o}{=}\PY{n+nb+bp}{F}\PY{p}{,}\PY{n}{order}\PY{o}{=}\PY{n+nf}{c}\PY{p}{(}\PY{l+m}{1}\PY{p}{,}\PY{l+m}{1}\PY{p}{)}\PY{p}{)}
\PY{n}{res}\PY{o}{\PYZlt{}\PYZhy{}}\PY{n}{sp}\PY{o}{\PYZdl{}}\PY{n}{res}\PY{p}{[}\PY{l+m}{\PYZhy{}1}\PY{p}{]}
\PY{n+nf}{plot}\PY{p}{(}\PY{n}{res}\PY{p}{,}\PY{n}{type}\PY{o}{=}\PY{l+s}{\PYZdq{}}\PY{l+s}{l\PYZdq{}}\PY{p}{)}
\PY{n+nf}{acf}\PY{p}{(}\PY{n}{res}\PY{p}{)}
\PY{n+nf}{acf}\PY{p}{(}\PY{n}{res}\PY{o}{\PYZca{}}\PY{l+m}{2}\PY{p}{)}
\PY{n+nf}{confint}\PY{p}{(}\PY{n}{sp}\PY{p}{)}
\PY{n+nf}{par}\PY{p}{(}\PY{n}{mfrow} \PY{o}{=} \PY{n+nf}{c}\PY{p}{(}\PY{l+m}{1}\PY{p}{,} \PY{l+m}{1}\PY{p}{)}\PY{p}{)}
\PY{n+nf}{garch}\PY{p}{(}\PY{n}{DAX}\PY{p}{,}\PY{n}{order}\PY{o}{=}\PY{n+nf}{c}\PY{p}{(}\PY{l+m}{1}\PY{p}{,}\PY{l+m}{1}\PY{p}{)}\PY{p}{,} \PY{n}{trace}\PY{o}{=}\PY{n+nb+bp}{F}\PY{p}{)}
\end{Verbatim}
\end{tcolorbox}

    A matrix: 3 × 2 of type dbl
\begin{tabular}{r|ll}
  & 2.5 \% & 97.5 \%\\
\hline
	a0 & 3.157588e-06 & 6.120989e-06\\
	a1 & 4.627785e-02 & 9.037965e-02\\
	b1 & 8.566877e-01 & 9.214456e-01\\
\end{tabular}


    
    
    \begin{Verbatim}[commandchars=\\\{\}]

Call:
garch(x = DAX, order = c(1, 1), trace = F)

Coefficient(s):
       a0         a1         b1  
4.639e-06  6.833e-02  8.891e-01  

    \end{Verbatim}

    
    \begin{center}
    \adjustimage{max size={0.9\linewidth}{0.9\paperheight}}{output_101_2.png}
    \end{center}
    { \hspace*{\fill} \\}
    
    \begin{tcolorbox}[breakable, size=fbox, boxrule=1pt, pad at break*=1mm,colback=cellbackground, colframe=cellborder]
\prompt{In}{incolor}{452}{\boxspacing}
\begin{Verbatim}[commandchars=\\\{\}]
\PY{n+nf}{par}\PY{p}{(}\PY{n}{mfrow} \PY{o}{=} \PY{n+nf}{c}\PY{p}{(}\PY{l+m}{3}\PY{p}{,} \PY{l+m}{1}\PY{p}{)}\PY{p}{)}
\PY{n}{sp}\PY{o}{\PYZlt{}\PYZhy{}}\PY{n+nf}{garch}\PY{p}{(}\PY{n}{SMI}\PY{p}{,} \PY{n}{trace}\PY{o}{=}\PY{n+nb+bp}{F}\PY{p}{)}
\PY{n}{res}\PY{o}{\PYZlt{}\PYZhy{}}\PY{n}{sp}\PY{o}{\PYZdl{}}\PY{n}{res}\PY{p}{[}\PY{l+m}{\PYZhy{}1}\PY{p}{]}
\PY{n+nf}{plot}\PY{p}{(}\PY{n}{res}\PY{p}{,}\PY{n}{type}\PY{o}{=}\PY{l+s}{\PYZdq{}}\PY{l+s}{l\PYZdq{}}\PY{p}{)}
\PY{n+nf}{acf}\PY{p}{(}\PY{n}{res}\PY{p}{)}
\PY{n+nf}{acf}\PY{p}{(}\PY{n}{res}\PY{o}{\PYZca{}}\PY{l+m}{2}\PY{p}{)}
\PY{n+nf}{confint}\PY{p}{(}\PY{n}{sp}\PY{p}{)}
\PY{n+nf}{par}\PY{p}{(}\PY{n}{mfrow} \PY{o}{=} \PY{n+nf}{c}\PY{p}{(}\PY{l+m}{1}\PY{p}{,} \PY{l+m}{1}\PY{p}{)}\PY{p}{)}
\PY{n+nf}{garch}\PY{p}{(}\PY{n}{SMI}\PY{p}{,}\PY{n}{order}\PY{o}{=}\PY{n+nf}{c}\PY{p}{(}\PY{l+m}{1}\PY{p}{,}\PY{l+m}{1}\PY{p}{)}\PY{p}{,} \PY{n}{trace}\PY{o}{=}\PY{n+nb+bp}{F}\PY{p}{)}
\end{Verbatim}
\end{tcolorbox}

    A matrix: 3 × 2 of type dbl
\begin{tabular}{r|ll}
  & 2.5 \% & 97.5 \%\\
\hline
	a0 & 8.974129e-06 & 1.441805e-05\\
	a1 & 7.640451e-02 & 1.525530e-01\\
	b1 & 6.899628e-01 & 8.146311e-01\\
\end{tabular}


    
    
    \begin{Verbatim}[commandchars=\\\{\}]

Call:
garch(x = SMI, order = c(1, 1), trace = F)

Coefficient(s):
       a0         a1         b1  
0.0000117  0.1144787  0.7522969  

    \end{Verbatim}

    
    \begin{center}
    \adjustimage{max size={0.9\linewidth}{0.9\paperheight}}{output_102_2.png}
    \end{center}
    { \hspace*{\fill} \\}
    
    \begin{tcolorbox}[breakable, size=fbox, boxrule=1pt, pad at break*=1mm,colback=cellbackground, colframe=cellborder]
\prompt{In}{incolor}{425}{\boxspacing}
\begin{Verbatim}[commandchars=\\\{\}]
\PY{n+nf}{par}\PY{p}{(}\PY{n}{mfrow} \PY{o}{=} \PY{n+nf}{c}\PY{p}{(}\PY{l+m}{3}\PY{p}{,} \PY{l+m}{1}\PY{p}{)}\PY{p}{)}
\PY{n}{sp}\PY{o}{\PYZlt{}\PYZhy{}}\PY{n+nf}{garch}\PY{p}{(}\PY{n}{CAC}\PY{p}{,} \PY{n}{trace}\PY{o}{=}\PY{n+nb+bp}{F}\PY{p}{)}
\PY{n}{res}\PY{o}{\PYZlt{}\PYZhy{}}\PY{n}{sp}\PY{o}{\PYZdl{}}\PY{n}{res}\PY{p}{[}\PY{l+m}{\PYZhy{}1}\PY{p}{]}
\PY{n+nf}{plot}\PY{p}{(}\PY{n}{res}\PY{p}{,}\PY{n}{type}\PY{o}{=}\PY{l+s}{\PYZdq{}}\PY{l+s}{l\PYZdq{}}\PY{p}{)}
\PY{n+nf}{acf}\PY{p}{(}\PY{n}{res}\PY{p}{)}
\PY{n+nf}{acf}\PY{p}{(}\PY{n}{res}\PY{o}{\PYZca{}}\PY{l+m}{2}\PY{p}{)}
\PY{n+nf}{confint}\PY{p}{(}\PY{n}{sp}\PY{p}{)}
\PY{n+nf}{par}\PY{p}{(}\PY{n}{mfrow} \PY{o}{=} \PY{n+nf}{c}\PY{p}{(}\PY{l+m}{1}\PY{p}{,} \PY{l+m}{1}\PY{p}{)}\PY{p}{)}
\PY{n+nf}{garch}\PY{p}{(}\PY{n}{CAC}\PY{p}{,}\PY{n}{order}\PY{o}{=}\PY{n+nf}{c}\PY{p}{(}\PY{l+m}{1}\PY{p}{,}\PY{l+m}{1}\PY{p}{)}\PY{p}{,} \PY{n}{trace}\PY{o}{=}\PY{n+nb+bp}{F}\PY{p}{)}
\end{Verbatim}
\end{tcolorbox}

    A matrix: 3 × 2 of type dbl
\begin{tabular}{r|ll}
  & 2.5 \% & 97.5 \%\\
\hline
	a0 & 6.542420e-06 & 1.702857e-05\\
	a1 & 3.712088e-02 & 8.119886e-02\\
	b1 & 7.821879e-01 & 9.056679e-01\\
\end{tabular}


    
    
    \begin{Verbatim}[commandchars=\\\{\}]

Call:
garch(x = CAC, order = c(1, 1), trace = F)

Coefficient(s):
       a0         a1         b1  
1.179e-05  5.916e-02  8.439e-01  

    \end{Verbatim}

    
    \begin{center}
    \adjustimage{max size={0.9\linewidth}{0.9\paperheight}}{output_103_2.png}
    \end{center}
    { \hspace*{\fill} \\}
    
    \begin{tcolorbox}[breakable, size=fbox, boxrule=1pt, pad at break*=1mm,colback=cellbackground, colframe=cellborder]
\prompt{In}{incolor}{426}{\boxspacing}
\begin{Verbatim}[commandchars=\\\{\}]
\PY{n+nf}{par}\PY{p}{(}\PY{n}{mfrow} \PY{o}{=} \PY{n+nf}{c}\PY{p}{(}\PY{l+m}{3}\PY{p}{,} \PY{l+m}{1}\PY{p}{)}\PY{p}{)}
\PY{n}{sp}\PY{o}{\PYZlt{}\PYZhy{}}\PY{n+nf}{garch}\PY{p}{(}\PY{n}{FTSE}\PY{p}{,} \PY{n}{trace}\PY{o}{=}\PY{n+nb+bp}{F}\PY{p}{)}
\PY{n}{res}\PY{o}{\PYZlt{}\PYZhy{}}\PY{n}{sp}\PY{o}{\PYZdl{}}\PY{n}{res}\PY{p}{[}\PY{l+m}{\PYZhy{}1}\PY{p}{]}
\PY{n+nf}{plot}\PY{p}{(}\PY{n}{res}\PY{p}{,}\PY{n}{type}\PY{o}{=}\PY{l+s}{\PYZdq{}}\PY{l+s}{l\PYZdq{}}\PY{p}{)}
\PY{n+nf}{acf}\PY{p}{(}\PY{n}{res}\PY{p}{)}
\PY{n+nf}{acf}\PY{p}{(}\PY{n}{res}\PY{o}{\PYZca{}}\PY{l+m}{2}\PY{p}{)}
\PY{n+nf}{confint}\PY{p}{(}\PY{n}{sp}\PY{p}{)}
\PY{n+nf}{par}\PY{p}{(}\PY{n}{mfrow} \PY{o}{=} \PY{n+nf}{c}\PY{p}{(}\PY{l+m}{1}\PY{p}{,} \PY{l+m}{1}\PY{p}{)}\PY{p}{)}
\PY{n+nf}{garch}\PY{p}{(}\PY{n}{FTSE}\PY{p}{,}\PY{n}{order}\PY{o}{=}\PY{n+nf}{c}\PY{p}{(}\PY{l+m}{1}\PY{p}{,}\PY{l+m}{1}\PY{p}{)}\PY{p}{,} \PY{n}{trace}\PY{o}{=}\PY{n+nb+bp}{F}\PY{p}{)}
\end{Verbatim}
\end{tcolorbox}

    A matrix: 3 × 2 of type dbl
\begin{tabular}{r|ll}
  & 2.5 \% & 97.5 \%\\
\hline
	a0 & 2.678668e-07 & 1.476560e-06\\
	a1 & 3.201160e-02 & 5.863026e-02\\
	b1 & 9.219037e-01 & 9.618274e-01\\
\end{tabular}


    
    
    \begin{Verbatim}[commandchars=\\\{\}]

Call:
garch(x = FTSE, order = c(1, 1), trace = F)

Coefficient(s):
       a0         a1         b1  
8.722e-07  4.532e-02  9.419e-01  

    \end{Verbatim}

    
    \begin{center}
    \adjustimage{max size={0.9\linewidth}{0.9\paperheight}}{output_104_2.png}
    \end{center}
    { \hspace*{\fill} \\}
    
    From the correlograms of the residuals we can conclude that the
residuals are uncorrelated, thus being IID noise. We can also see that
the correlograms of the squares are also IID noise


    % Add a bibliography block to the postdoc
    
    
    
\end{document}
